\documentclass[10pt,letterpaper]{article}
\usepackage[utf8]{inputenc}
\usepackage{amsmath,amsthm,amsfonts,amssymb}

\newcommand{\X}{\ensuremath{\mathcal{X}}}
\newcommand{\R}{\ensuremath{\mathcal{R}}}
\newcommand{\mP}{\ensuremath{\mathcal{P}}}
\newcommand{\mM}{\ensuremath{\mathcal{M}}}
\newcommand{\ip}[2]{\ensuremath{\langle #1, #2 \rangle}}
\newcommand{\grad}{\nabla}
\newcommand{\one}{\ensuremath{\mathbf{1}}}
\newcommand{\co}{\mbox{co}}

\newcommand{\calL}{{\mathcal{L}}}
\newcommand{\scr}{{\text{SC}}}
\newcommand{\mm}{\text{MM}}

\newcommand{\rank}{{\calL(C)}}
\newcommand{\uni}{{\rank^n}}
\newcommand{\sca}{{\scr^{\alpha}}}
\newcommand{\sort}{\text{SORT}}
\newcommand{\phia}{\phi^{\alpha}}
\newcommand{\mmphia}{\mm^{\phia}}

\newcommand{\muhat}{\hat{\mu}}
\newcommand{\that}{\hat{\theta}}

\DeclareMathOperator*{\argmax}{arg\,max}
\DeclareMathOperator*{\argmin}{arg\,min}
\newcommand{\eps}{\epsilon}

\newtheorem{theorem}{Theorem}
\newtheorem{lemma}{Lemma}
\newtheorem{conjecture}{Conjecture}
\newtheorem{claim}{Claim}
\newtheorem{proposition}{Proposition}


\title{Euclidean Voting}

\begin{document}
\maketitle

%%%
%%%
%%%

\section{Simple Ranking Scoring Functions / Generalized Scoring Rules and Mean Proximity Rules}

These are defined by both Conitzer et. al. and Zwicker. 
\begin{itemize}
\item SRSF: $\argmax_{\sigma} \sum_{i=1}^n s(\sigma,\sigma_i)$
\item GSR: $\argmax_{o} \sum_{i=1}^n s(o,\sigma_i)$ where $o \in \mathcal{O}$ can be from any outcome space. Restricting outcome space to rankings and alternatives recover SRSF and SCSF, respectively.
\item Mean Proximity: Embedding $\phi$ of votes, another embedding $\psi$ of outcomes, select the outcome whose $\psi$ is close to the mean $\phi$ of input votes. Restricting outcomes to rankings still leaves open possibility of using $\psi \neq \phi$. 
\item THM: Mean Proximity $\Leftrightarrow$ GSR. So, Mean proximity for rankings $\Leftrightarrow$ SRSF (GSR for rankings)
\item {\bf OUR THM:} Mean proximity for rankings with $\psi = \phi$ $\Leftrightarrow$ SRSF with symmetric $s$. 
	\begin{itemize}
	\item Have a coordinate for every pair of rankings rather than each ranking. The coordinate for $\sigma,\sigma'$ in embedding of $\sigma$ has value $\sqrt{s(\sigma,\sigma')}$. Coordinates for all other $\sigma',\sigma''$ are $0$. 
	\end{itemize}
\end{itemize}

Properties:
\begin{itemize}
\item Neutral iff $s = $ neutral. 
\item Neutral $\Rightarrow$ SRSF iff MLE
\item Consistent $+$ continuous (which characterize all \emph{anonymous mean neat voting rules})
\item Conjecture (Conitzer et. al.): Consistent $+$ continuous $+$ neutral $\Leftrightarrow$ Neutral SRSF
\item Capture PSR + KEM. But cannot capture Bucklin, Copeland, Maximin, Ranked Pairs (not consistent). STV is also not SRSF. 
\end{itemize}

\section{Linear Mean Proximity Rules}

Our work: Achieving neutrality by not-so-restricting linear representations.
\begin{itemize}
\item Definition
\item Motivation: While SRSF capture many nice voting rules, they capture bad ones too. Don't have $s(\sigma,\sigma')$ maximized by $\sigma' = \sigma$ then will not even satisfy strong unanimity. Linearity is unrestrictive, structural, and easy to work with. 
\item Can now define distance function, with nice properties (left invariance). Can observe that KT distance = Euclidean distance square. So Kemeny is a mean rule. So are all positional scoring rules. 
	\begin{itemize}
	\item {\bf Question:} Distance (and hence distance square) = MC/PC $\Rightarrow$ implications on the properties of the rule?
	\item For other, we need to check if embedding of symmetric SRSF that is in addition neutral is linear. 
	\end{itemize}
\item Still capture PSR+KEM. Of course cannot capture rules that are not SRSF. 
\item {\bf Question:} Captures all ``pairwise comparison scoring rules''?
\item We know: symmetric SRSF iff mean proximity (with same embedding). {\bf Our Conjecture:} Neutral SRSF iff linear mean proximity (with same embedding)?
	\begin{itemize}
	\item Neutrality $\Rightarrow$ symmetric. Hence, one direction is clear.
	\item For other, we need to check if embedding of symmetric SRSF that is in addition neutral is linear. 
	\end{itemize}
\end{itemize}

\section{Connections to other approaches}
\subsection{Axiomatic}
Consistency, continuity, anonymity, neutrality already described above. 

{\bf Question:} When is it PM-c, monotonic, majority for rankings?

\subsection{DR}
Consensus = Strong unanimity. Distance = Square of Euclidean distance. Votewise DR rules. 

Square is not always a distance metric $\rightarrow$ Bad. More meaningfully, define votewise DR rules by sum of squares of distances rather than sum of distances. . 

\subsection{MLE}
Take $\Pr[\sigma | \sigma^*] \propto e^{\|\phi(\sigma)-\phi(\sigma*)\|^2}$. Not always Mallows since square is not always a metric. But actually better $\rightarrow$ Gaussian. 

Neutrality $\Rightarrow$ Normalization independent of the true ranking $\Rightarrow$ Linear mean proximity rule is the MLE for this model. 

{\bf Question:} Efficient sampling?

{\bf Question:} Anything interesting for the Gaussian distributions that connect to PSR?

\section{Other Directions}

What are the equivalence classes of $\phi$ that lead to the same voting rule?

What about notions of consensus in Euclidean spaces other than mean $\rightarrow$ e.g., minimize maximum distance (equally let go)?
\end{document}