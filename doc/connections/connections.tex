\documentclass[10pt,letterpaper]{article}
\usepackage[utf8]{inputenc}
\usepackage{amsmath,amsthm,amsfonts,amssymb}
\usepackage{verbatim}
\usepackage[demo]{graphicx}
\usepackage{subfigure}

\newcommand{\X}{\ensuremath{\mathcal{X}}}
\newcommand{\R}{\ensuremath{\mathcal{R}}}
\newcommand{\mP}{\ensuremath{\mathcal{P}}}
\newcommand{\mM}{\ensuremath{\mathcal{M}}}
\newcommand{\ip}[2]{\ensuremath{\langle #1, #2 \rangle}}
\newcommand{\grad}{\nabla}
\newcommand{\one}{\ensuremath{\mathbf{1}}}
\newcommand{\co}{\mbox{co}}

\newcommand{\calL}{{\mathcal{L}}}
\newcommand{\rank}{{\calL(A)}}
\newcommand{\calO}{{\mathcal{O}}}

\newcommand{\uni}{{\rank^n}}
\newcommand{\sca}{{\scr^{\alpha}}}
\newcommand{\sort}{\text{SORT}}
\newcommand{\phia}{\phi^{\alpha}}
\newcommand{\mmphia}{\mm^{\phia}}

\newcommand{\muhat}{\hat{\mu}}
\newcommand{\that}{\hat{\theta}}

\DeclareMathOperator*{\argmax}{arg\,max}
\DeclareMathOperator*{\argmin}{arg\,min}
\newcommand{\eps}{\epsilon}

\newtheorem{theorem}{Theorem}
\newtheorem{lemma}{Lemma}
\newtheorem{conjecture}{Conjecture}
\newtheorem{claim}{Claim}
\newtheorem{proposition}{Proposition}


\newenvironment{definition}[1][Definition]{\begin{trivlist}
\item[\hskip \labelsep {\bfseries #1}]}{\end{trivlist}}
\newenvironment{example}[1][Example]{\begin{trivlist}
\item[\hskip \labelsep {\bfseries #1}]}{\end{trivlist}}
\newenvironment{remark}[1][Remark]{\begin{trivlist}
\item[\hskip \labelsep {\bfseries #1}]}{\end{trivlist}}


\title{Euclidean Voting}

\begin{document}
\maketitle

%%%
%%%
%%%

\section{Properties of Voting Rules}

Neutrality, Consistency, Connectedness

\section{Generalized Scoring Rules and Mean Proximity Rules}
Let $\rank$ and $\calO$ denote the space of rankings and outcomes, respectively. Given any profile $\pi$, let $n(\pi,\sigma)$ denote the number of times $\sigma$ appears in $\pi$. Let $\sigma_1,\ldots,\sigma_{m!}$ denote a fixed reference order of the rankings in $\rank$. 

\begin{definition}[Generalized Scoring Rules (Zwicker~\cite{Zwicker08a})]
A voting rule is called a generalized scoring rule if there exists a scoring function $s : \rank \times \calO \rightarrow \mathbb{R}$ such that given any profile $\pi$, the rule returns the outcome $\argmax_{o \in \calO} \sum_{\sigma \in \rank} n(\pi,\sigma) \cdot s(\sigma,o)$. 
\end{definition}

\begin{definition}[Mean Proximity Rules (Zwicker~\cite{Zwicker08a})]
A voting rule is called a mean proximity rule if there exists an input embedding $\phi : \rank \rightarrow \mathbb{R}^k$ and an output embedding $\psi: \calO \rightarrow \mathbb{R}^k$ such that given any profile $\pi$ with $n$ votes, the rule returns the outcome $\argmin_{o \in \calO} \|\psi(o) - mean(\pi) \|$, where $mean(\pi) = \sum_{\sigma \in \rank} (n(\pi,\sigma)/n) \cdot \phi(\sigma)$ is the mean of the input embeddings of the votes in $\pi$ (together with their multiplicity). 
\end{definition}

\begin{proposition}[Theorem 4.2.1, Zwicker~\cite{Zwicker08a}]
A voting rule is a generalized scoring rule if and only if it is a mean proximity rule.
\label{prop:equiv}
\end{proposition}

Further, Zwicker~\cite{Zwicker08b} shows that the set of voting rules that are \emph{consistent} and \emph{connected} is identical to the set of mean neat voting rules, which is a generalization of mean proximity rules. As a simple corollary, we have the following.

\begin{proposition}
Any mean proximity rule (by Proposition~\ref{prop:equiv}, equivalently, any generalized scoring rule) is consistent and connected.
\end{proposition}

This implies that any voting rule that is not consistent (in the SWF sense) is not a mean proximity rule. In fact, we have the following.

\begin{lemma}[Conitzer et. al.~\cite{CRX09}]
All positional scoring rules and the Kemeny rule are generalized scoring rules (i.e., mean proximity rules). However, Bucklin's rule, Copeland's rule, the maximin rule, and the ranked pairs method are not generalized scoring rules since they do not satisfy consistency. While STV does satisfy consistency, it is also not a generalized scoring rule. 
\end{lemma}

We are interested in social welfare functions (SWFs) that return a ranking, so $\calO = \rank$. In this case, the scoring function $s : \rank \times \rank \rightarrow \mathbb{R}$ describes the \emph{similarity} between two rankings. This special case was rediscovered by Conitzer et. al.~\cite{CRX09} and named \emph{simple ranking scoring functions} (SRSFs). 

Note that in this case, the rule can be encoded by a score matrix $S$ such that $S_{ij} = s(\sigma_i,\sigma_j)$. Recall that $\sigma_1,\ldots,\sigma_{m!}$ is any fixed reference order of the rankings. Now, given any profile $\pi$, we can create a vector $y_{\pi} = [n(\pi,\sigma_1) \ldots n(\pi,\sigma_{m!})]^T$. Then, it is easy to verify that the output of the rule would be $\sigma_k$ such that $k = \argmax_i (S*y_{\pi})_i$, that is, the ranking such that its associated coordinate in $S*y_{\pi}$ is maximum. From here onwards, unless mentioned otherwise, we will consider only social welfare functions. 

\subsection{Symmetric Mean Proximity Rules}
In the case of social welfare functions, since the outcome space is identical to the input space, it makes sense to have the input embedding identical to the output embedding, i.e., $\phi = \psi$. Indeed, the well known mean proximity rules such as all positional scoring rules (when interpreted as SWFs) and the Kemeny rule can be achieved with $\phi = \psi$. We call such rules symmetric mean proximity rules.

\begin{definition}[Symmetric Mean Proximity Rules]
A voting rule is called symmetric mean proximity rule if there exists a mean proximity representation of the rule where the input embedding $\phi$ and the output embedding $\psi$ satisfy $\phi = \psi$. 
\end{definition} 

We now show a characterization of symmetric mean proximity rules along the lines of Proposition~\ref{prop:equiv}. 
\begin{theorem}
A voting rule is a symmetric mean proximity rule if and only if it is a generalized scoring rule whose score matrix is positive semidefinite (PSD).\footnote{From the proof of Theorem~\ref{thm:symm}, it is clear that some score matrix representing a generalized scoring rule is PSD if and only if every score matrix representing the rule is PSD.}
\label{thm:symm}
\end{theorem}
\begin{proof}
First, 
\end{proof}

\begin{conjecture}
Let $S$ and $S'$ be two score matrices with rows $\{r_i\}$ and $\{r'_i\}$. Then, $S$ and $S'$ correspond to the same generalized scoring rule if and only if there exist $a \in \mathbb{R}$ and $b \in \mathbb{R}^{m!}$ such that for every $i$, $r'_i = a \cdot r_i + b$. 
\end{conjecture}

\section{Neutrality in Symmetric Mean Proximity Rules and Linear Embeddings}

Conitzer et. al.~\cite{CRX09} showed that neutrality of a GSR is equivalent to neutrality of its scoring function.
\begin{proposition}[Lemma 2, Conitzer et. al.~\cite{CRX09}]
A generalized scoring rule is neutral if and only if it corresponds to a neutral scoring function $s : \rank \times \rank \rightarrow \mathbb{R}$ where $s(\tau \circ \sigma,\tau \circ \sigma') = s(\sigma,\sigma')$ for every $\tau, \sigma, \sigma' \in \rank$ and $\circ$ is the composition when rankings are seen as bijective functions, or equivalently the product operator of the symmetric group. 
\end{proposition}

\begin{definition}[Linear Embeddings]
$\phi(\tau \circ \sigma) = R_{\tau} \cdot \phi(\sigma)$. 
\end{definition}

Motivation: While mean proximity rules / generalized scoring rules capture many nice voting rules, they capture bad ones too. In particular, if we have $s(\sigma,\sigma') > s(\sigma,\sigma)$ for any $\sigma$ and $\sigma'$, then the rule will not return $\sigma$ on the profile where all votes are $\sigma$, thus violating strong unanimity. While linear embeddings solve this problem and achieve many other desirable properties, they are quite unrestrictive - they still capture all positional scoring rules and the Kemeny rule.

\begin{theorem}
Any mean proximity rule that has a representation using a linear embedding satisfies strong unanimity and neutrality. 
\end{theorem}

In addition, we can now define a distance function $d(\sigma,\sigma') = \|\phi(\sigma)-\phi(\sigma')\|$. It is easy to show that this, in addition to being a distance metric, is also left-invariant. Additionally, if we take the linear embedding that generates the Kemeny rule, then the corresponding distance function becomes the squaure root of the KT distance. This demonstrates that the KT distance is actually the square of a more natural Euclidean distance metric. In this sense, the Kemeny rule is a mean rule, rather than a median rule, and so are all positional scoring rules. 

\begin{theorem}
A symmetric mean proximity rule is neutral if and only if there exists a representation $\phi$ of the rule that is linear. 
\label{thm:neutrality-linear-embedding}
\end{theorem}

\section{Euclidean Embeddings and Dimensions}

In this section, we analyze the following general question. \emph{Given a symmetric mean proximity rule $r$, what is the minimum dimension of any embedding that represents $r$?}. 

\begin{theorem}
The minimum dimension required for any positional scoring rule is $m-1$. 
\end{theorem}

\subsection{Borda Rule}
In this section, we analyze the $m-1$ dimension embeddings of the Borda rule. 

{\bf $\mathbf{3}$ alternatives:} Zwicker~\cite{Zwicker08a} informally demonstrated that for $3$ alternatives, any embedding that represents the Borda rule embeds the $6$ possible rankings at the corners of a regular hexagon as shown in Figure~\ref{fig:borda-3alt}. In fact, it is exactly the \emph{permutahedron} of the symmetric group $S_3$ where each ranking is connected to two rankings that are obtained by two possible adjacent swaps. Thus, neighbouring rankings have \emph{Kendall Tau} distance $1$ from each other. Note that there are hexagons that connect rankings at KT distance greater than $1$, but those do not correspond to the Borda rule.

\begin{figure}
\centering
\begin{subfigure}[$3$ alternatives]
  %\includegraphics[width=.4\linewidth]{Borda3}
  {\rule{3cm}{3cm}}
  \label{fig:borda-3alt}
\end{subfigure}%
\begin{subfigure}[$4$ alternatives]
  %\includegraphics[width=.4\linewidth]{Borda4}
  {\rule{3cm}{3cm}}
  \label{fig:borda-4alt}
\end{subfigure}%
\caption{Embeddings for the Borda Rule}
\label{fig:borda}
\end{figure}

{\bf $\mathbf{4}$ alternatives:} Interestingly, the observation that the Borda rule embeds rankings to the vertices of the regular polytope of the permutahedron carries over to the case of $4$ alternatives. Thus, each ranking is connected to $3$ rankings that are obtained by performing one of the three possible adjacent swaps. The polytope of the permutahedron of $S_4$, shown in Figure~\ref{fig:borda-4alt}, consists of $8$ hexagons and $6$ squares. Each hexagon contains the $6$ rankings that are obtained by either keeping the first alternative constant or the last alternative constant. Note that we can forget the alternative that is constant, and the hexagon exactly matches the embeddings of the rankings over the remaining $3$ alternatives. The $6$ squares each contain $4$ rankings that are obtained by swapping the first two or the last two alternatives. Thus, each square has the following form: 
$$
a \succ b \succ c \succ d \;\longrightarrow\; a \succ b \succ d \succ c \;\longrightarrow\; b \succ a \succ d \succ c \;\longrightarrow\; b \succ a \succ c \succ d.
$$
{\bf Replace this by a figure of a square, similarly draw a hexagon}

For better understanding of permutahedrons, refer to Crisman~\cite{Crisman}. 


\section{Connections to other approaches}
\subsection{Axiomatic}
Consistency, continuity, anonymity, neutrality already described above. 

{\bf Question:} When is it PM-c (related to the Euclidean distance being MC?), monotonic, majority for rankings?

\subsection{DR}
Consensus = Strong unanimity. Distance = Square of Euclidean distance. Votewise DR rules. 

Square is not always a distance metric $\rightarrow$ Bad. More meaningfully, define votewise DR rules by sum of squares of distances rather than sum of distances. . 

\subsection{MLE}
Take $\Pr[\sigma | \sigma^*] \propto e^{\|\phi(\sigma)-\phi(\sigma*)\|^2}$. Not always Mallows since square is not always a metric. But actually better $\rightarrow$ Gaussian. 

Neutrality $\Rightarrow$ Normalization independent of the true ranking $\Rightarrow$ Linear mean proximity rule is the MLE for this model. 

{\bf Question:} Efficient sampling?

{\bf Question:} Anything interesting for the Gaussian distributions that connect to PSR?

\section{Research Questions}
\begin{enumerate}
\item What shape, and what corresponding voting rule do we get if we replace the KT distance by some other distance in the permutahedron?
\item Proving a lower bound on the dimensions for the Kemeny rule
\item Conjecture by Conitzer et. al.~\cite{CRX09}: Consistent $+$ continuous $+$ neutral $\Leftrightarrow$ Neutral SRSF
\item What are the equivalence classes of $\phi$ that lead to the same voting rule?
\item What about notions of consensus in Euclidean spaces other than mean $\rightarrow$ e.g., minimize maximum distance (everyone ``lets go'' equally)?
\end{enumerate}


\begin{comment}
Mean proximity rule / generalized scoring rule / SRSF - Neutral $\Rightarrow$ SRSF iff MLE
{\bf Question:} (Linear) Mean Proximity Rules - Captures all ``pairwise comparison scoring rules''?
\end{comment}

\bibliographystyle{plain}
\bibliography{abbshort,ultimate}
\end{document}