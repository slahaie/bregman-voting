\documentclass[10pt,letterpaper]{article}
\usepackage[utf8]{inputenc}
\usepackage[T1]{fontenc}
\usepackage{lmodern}
\usepackage{environ}
\usepackage{color}
\usepackage{amsmath,amsthm,amsfonts,amssymb}
\usepackage{verbatim}
\usepackage[demo]{graphicx}
\usepackage{subfigure}

%%%%%%%%%%%%%%%%%%%%%%%%%%%%%%%%%%%%%%%%%%%%%
%%%%%%%%%%%%%%%%%%%%%%%%%%%%%%%%%%%%%%%%%%%%%
%%%%%%%%%%%%%%%%%%%%%%%%%%%%%%%%%%%%%%%%%%%%%

\newcommand{\X}{\ensuremath{\mathcal{X}}}
\newcommand{\R}{\ensuremath{\mathcal{R}}}
\newcommand{\mP}{\ensuremath{\mathcal{P}}}
\newcommand{\mM}{\ensuremath{\mathcal{M}}}
\newcommand{\ip}[2]{\ensuremath{\langle #1, #2 \rangle}}
\newcommand{\grad}{\nabla}
\newcommand{\one}{\ensuremath{\mathbf{1}}}
\newcommand{\co}{\mbox{co}}

\newcommand{\calL}{{\mathcal{L}}}
\newcommand{\rank}{{\calL(A)}}
\newcommand{\calO}{{\mathcal{O}}}
\newcommand{\calP}{{\mathcal{P}}}

\newcommand{\uni}{{\rank^n}}
\newcommand{\sca}{{\scr^{\alpha}}}
\newcommand{\sort}{\text{SORT}}
\newcommand{\phia}{\phi^{\alpha}}
\newcommand{\mmphia}{\mm^{\phia}}

\newcommand{\muhat}{\hat{\mu}}
\newcommand{\that}{\hat{\theta}}

\DeclareMathOperator*{\argmax}{arg\,max}
\DeclareMathOperator*{\argmin}{arg\,min}
\newcommand{\eps}{\epsilon}

\newtheorem{theorem}{Theorem}
\newtheorem{lemma}{Lemma}
\newtheorem{conjecture}{Conjecture}
\newtheorem{claim}{Claim}
\newtheorem{proposition}{Proposition}
\newtheorem{corollary}{Corollary}

\newenvironment{definition}[1][Definition]{\begin{trivlist}
\item[\hskip \labelsep {\bfseries #1}]}{\end{trivlist}}
\newenvironment{example}[1][Example]{\begin{trivlist}
\item[\hskip \labelsep {\bfseries #1}]}{\end{trivlist}}
\newenvironment{remark}[1][Remark]{\begin{trivlist}
\item[\hskip \labelsep {\bfseries #1}]}{\end{trivlist}}


\newcount\Comments
\Comments=1
\newcommand{\kibitz}[2]{\ifnum\Comments=1\textcolor{#1}{#2}\fi}
\newcommand{\csl}[1]{\kibitz{blue} {[SL: #1]}}
\newcommand{\cns}[1]{\kibitz{red} {[NS: #1]}}

\newcommand{\nt}{NT}

%%%%%%%%%%%%%%%%%%%%%%%%%%%%%%%%%%%%%%%%%%%%%
%%%%%%%%%%%%%%%%%%%%%%%%%%%%%%%%%%%%%%%%%%%%%
%%%%%%%%%%%%%%%%%%%%%%%%%%%%%%%%%%%%%%%%%%%%%

\title{Euclidean Voting}
\author{S\'{e}bastien Lahaie\\Microsoft Research New York, USA\\slahaie@microsoft.com \and Nisarg Shah\\Carnegie Mellon University, USA\\nkshah@cs.cmu.edu}
\date{}
\begin{document}
\maketitle

%%%%%%%%%%%%%%%%%%%%%%%%%%%%%%%%%%%%%%%%%%%%%
%%%%%%%%%%%%%%%%%%%%%%%%%%%%%%%%%%%%%%%%%%%%%
%%%%%%%%%%%%%%%%%%%%%%%%%%%%%%%%%%%%%%%%%%%%%

\section{Preliminaries}

Let $A$ denote the set of $m$ alternatives, and $\rank$ and $\calO$ denote the space of rankings of alternatives and outcomes, respectively. Thus, $|\rank| = m!$, and let $k = |\calO|$. A profile $\pi \in \rank^n$ is a collection of votes (rankings). For any profile $\pi$, let $n(\pi,\sigma)$ denote the number of times $\sigma$ appears in $\pi$. Let $\sigma_1,\ldots,\sigma_{m!}$ denote a fixed reference order of the rankings in $\rank$. 

For any two profiles $\pi_1$ and $\pi_2$, let $\pi_1+\pi_2$ be the union profile such that $n(\pi_1+\pi_2,\sigma) = n(\pi_1,\sigma)+n(\pi_2,\sigma)$ for every $\sigma \in \rank$. Similarly, for any profile $\pi$, let $c \pi$ be the profile such that $n(c \pi,\sigma) = c \cdot n(\pi,\sigma)$ for every $\sigma \in \rank$. 

%%%%%%%%%%%%%%%%%%%%%%%%%%%%%%%%%%%%%%%%%%%%%

\begin{definition}[Voting Rule]
A \emph{voting rule} (more technically, a social welfare function - SWF) $r : \rank^n \rightarrow \calP(\rank)\setminus\{\emptyset\}$ is a function that maps every profile of votes to a set of tied rankings. 
\end{definition}

Note that a voting rule can never output an empty set.

%%%%%%%%%%%%%%%%%%%%%%%%%%%%%%%%%%%%%%%%%%%%%

\begin{definition}[Anonymity]
A voting rule $r$ is called \emph{anonymous} if it only depends on the number of times each ranking appears in the profile: for every profiles $\pi_1$ and $\pi_2$ such that $n(\pi_1,\sigma) = n(\pi_2,\sigma)$ for every $\sigma \in \rank$, $r(\pi_1) = r(\pi_2)$. 
\end{definition}

%%%%%%%%%%%%%%%%%%%%%%%%%%%%%%%%%%%%%%%%%%%%%

\begin{definition}[Rank-Distinguishability]
A voting rule $r$ is called \emph{rank-distinguishing} if it can distinguish between any two rankings: for every two rankings $\sigma, \sigma' \in \rank$ ($\sigma \neq \sigma'$), there exists a profile $\pi$ such that exactly one of $\sigma$ or $\sigma'$ is in $r(\pi)$. 
\end{definition}

In this paper, we only consider rank-distinguishing voting rules.

%%%%%%%%%%%%%%%%%%%%%%%%%%%%%%%%%%%%%%%%%%%%%

\begin{definition}[Unanimity]
A voting rule $r$ is said to satisfy \emph{unanimity} if on every profile that consists of copies of a single ranking, it uniquely outputs that ranking: for every profile $\pi$ such that $n(\pi,\sigma) > 0$ for some $\sigma \in \rank$ and $n(\pi,\sigma') = 0$ for all $\sigma' \neq \sigma$, $r(\pi) = \{\sigma\}$. 
\end{definition}

%%%%%%%%%%%%%%%%%%%%%%%%%%%%%%%%%%%%%%%%%%%%%
\begin{definition}[Neutrality]
Given any profile $\pi = (\sigma_1,\ldots,\sigma_n)$, let $\tau \pi = (\tau \sigma_1,\ldots,\tau \sigma_n)$ be the profile where each vote is permuted according to $\tau$. Similarly, given any set of rankings $S$, let $\tau S = \{\tau \sigma | \sigma \in S\}$. A voting rule $r$ is called \emph{neutral} if for every profile $\pi$ and permutation $\tau$, we have $r(\tau \pi) = \tau r(\pi)$. 
\end{definition}

%%%%%%%%%%%%%%%%%%%%%%%%%%%%%%%%%%%%%%%%%%%%%

\begin{definition}[Consistency] %and Weak Consistency]
A social welfare function $r$ is called \emph{consistent} (for rankings) if for every profiles $\pi_1$ and $\pi_2$ such that $r(\pi_1) \cap r(\pi_2) \neq \emptyset$, $r(\pi_1+\pi_2) = r(\pi_1) \cap r(\pi_2)$. %A social welfare function $r$ is called weakly consistent (for rankings) if for every profiles $\pi_1$ and $\pi_2$ such that $r(\pi_1) = r(\pi_2)$, we have $r(\pi_1+\pi_2) = r(\pi_1) = r(\pi_2)$. 
\end{definition}

%%%%%%%%%%%%%%%%%%%%%%%%%%%%%%%%%%%%%%%%%%%%%

\begin{definition}[Connectedness]
A voting rule $r$ is called \emph{connected} if for any two profiles $\pi_1$ and $\pi_2$ with $r(\pi_1) \cap r(\pi_2) \neq \emptyset$, there exist non-negative integers $c$ and $d$ such that $r(\pi_1) \cap r(c \pi_1 + d \pi_2) \neq \emptyset$ and $r(\pi_1) \neq r(c \pi_1 + d \pi_2)$. 
\end{definition}

%%%%%%%%%%%%%%%%%%%%%%%%%%%%%%%%%%%%%%%%%%%%%

\cns{Check if this is equivalent/related to continuity defined by Conitzer et. al.~\cite{CRX09}. Then we can add that to Proposition~\ref{prop:properties}.}
\begin{definition}[Continuity]
Two profiles $\pi_1$ and $\pi_2$ satisfy $\pi_1 \approx \pi_2$ if they differ by one vote: for some $\sigma$ and $\sigma'$, $n(\pi_1,\sigma) = n(\pi_2,\sigma)-1$, $n(\pi_1,\sigma') = n(\pi_2,\sigma')+1$, and for every $\sigma \in \rank\setminus\{\sigma,\sigma'\}$, $n(\pi_1,\sigma) = n(\pi_2,\sigma)$. A voting rule $r$ is called \emph{continuous} if for every profile $\pi$ and ranking $\sigma$, $\sigma \notin r(\pi)$ implies that there exists integer $k$ such that for every profile $\pi' \approx k \pi$, $\sigma \notin r(\pi')$. 
\end{definition}

%%%%%%%%%%%%%%%%%%%%%%%%%%%%%%%%%%%%%%%%%%%%%
%%%%%%%%%%%%%%%%%%%%%%%%%%%%%%%%%%%%%%%%%%%%%
%%%%%%%%%%%%%%%%%%%%%%%%%%%%%%%%%%%%%%%%%%%%%

\section{Background on Mean Proximity Rules and Generalized Scoring Rules}

%%%%%%%%%%%%%%%%%%%%%%%%%%%%%%%%%%%%%%%%%%%%%

\begin{definition}[Mean Proximity Rules (Zwicker~\cite{Zwicker08a})]
A voting rule is called a \emph{mean proximity rule} if there exists an input embedding $\phi : \rank \rightarrow \mathbb{R}^k$ and an output embedding $\psi: \calO \rightarrow \mathbb{R}^k$ such that for any profile $\pi$ with $n$ votes, $r(\pi) = \argmin_{o \in \calO} \|\psi(o) - mean(\pi) \|$, where $mean(\pi) = (1/n) \cdot \sum_{\sigma \in \rank} n(\pi,\sigma) \cdot \phi(\sigma)$ is the mean of the input embeddings of the votes in $\pi$ (along with multiplicity). 
\end{definition}

%%%%%%%%%%%%%%%%%%%%%%%%%%%%%%%%%%%%%%%%%%%%%

\begin{definition}[Generalized Scoring Rules (Zwicker~\cite{Zwicker08a})]
A voting rule is called a \emph{generalized scoring rule} if there exists a scoring function $s : \rank \times \calO \rightarrow \mathbb{R}$ such that for any profile $\pi$, $r(\pi) = \argmax_{o \in \calO} \sum_{\sigma \in \rank} n(\pi,\sigma) \cdot s(\sigma,o)$. 
\end{definition}

%%%%%%%%%%%%%%%%%%%%%%%%%%%%%%%%%%%%%%%%%%%%%

%\begin{proposition}[Corollary~4.2.3, Zwicker~\cite{Zwicker08a}]
%Let $A = \{u_1,\ldots,u_l\}$ be a set of $l$ vectors in $\mathbb{R}^k$. Let $B = \{v_1,\ldots,v_p\}$ be a set of $p$ vectors in $\mathbb{R}^k$. Then, the discrete mean in $A$ of vectors in $B$ is the vector in $A$ that is closest to the Euclidean mean of vectors in $B$. That is,
%$$
%\argmin_{u_i \in A} \sum_{v_j \in B} \|u_i - v_j\|^2 = \argmin_{u_i \in A} \|u_i - (1/p) \cdot \sum_{v_j \in B} v_j\|.
%$$
%\label{prop:discrete-mean}
%\end{proposition}

For these two classes of voting rules, we have an elegant equivalence theorem by Zwicker~\cite{Zwicker08a}. The theorem uses an important result shown below.

\begin{proposition}[Zwicker~\cite{Zwicker08a}]
For any embeddings $\phi,\psi : \rank \rightarrow \mathbb{R}^k$ and any profile $\pi$,
$$
\argmin_{o \in \calO} \|\psi(o)-mean(\pi)\| = \argmin_{o \in \calO} \sum_{\sigma \in \rank} n(\pi,\sigma) \cdot \|\psi(o)-\phi(\sigma)\|^2.
$$ 
\label{prop:MPR-GSR-conversion}
\end{proposition}
\begin{proof}[Proof (reconstructed)]
\begin{align*}
&\argmin_{o \in \calO} \|\psi(o)-mean(\pi)\| \\
&\quad= \argmin_{o \in \calO} \|\psi(o)-mean(\pi)\|^2 \\
&\quad= \argmin_{o \in \calO} \|\psi(o)\|^2 + \|mean(\pi)\|^2 - 2 \cdot \langle \psi(o), (1/n) \sum_{\sigma \in \rank} n(\pi,\sigma) \phi(\sigma) \rangle\\
&\quad= \argmin_{o \in \calO} n \cdot \|\psi(o)\|^2 - 2 \sum_{\sigma \in \rank} n(\pi,\sigma) \cdot \langle \psi(o), \phi(\sigma) \rangle\\
&\quad= \argmin_{o \in \calO} \sum_{\sigma \in \rank} n(\pi,\sigma) \cdot (\|\psi(o)\|^2 - 2 \cdot \langle \psi(o), \phi(\sigma) \rangle )\\
&\quad= \argmin_{o \in \calO} \sum_{\sigma \in \rank} n(\pi,\sigma) \cdot (\|\psi(o)\|^2 - 2 \cdot \langle \psi(o), \phi(\sigma) \rangle + \|\phi(\sigma)\|^2)\\
&\quad= \argmin_{o \in \calO} \sum_{\sigma \in \rank} n(\pi,\sigma) \cdot \|\psi(o)-\phi(\sigma)\|^2.
\end{align*}
\end{proof}

From the definition of generalized scoring rules and Proposition~\ref{prop:MPR-GSR-conversion}, it is clear that the scoring function $s(\sigma,o) = -\|\psi(o)-\phi(\sigma)\|^2$ corresponds to the mean proximity rule given by $\psi$ and $\phi$. Zwicker~\cite{Zwicker08a} showed the following equivalence. 
\begin{proposition}[Theorem 4.2.1, Zwicker~\cite{Zwicker08a}]
A voting rule is a mean proximity rule if and only if it is a generalized scoring rule.
\label{prop:equiv}
\end{proposition}
%\begin{comment}
%\begin{proof}[Proof (reconstructed)]
%Take any mean proximity rule $r$. Let $\phi$ and $\psi$ be any input and output embeddings that generate $r$. Using Proposition~\ref{prop:MPR-GSR-conversion}, we can see that $r$ is a generalized scoring rule with the score function $s(\sigma,o) = -\|\psi(o)-\phi(\sigma)\|^2$. 
%
%For the other direction, take any generalized scoring rule $r$ and let $s$ be any score function that generates $r$. Then, let $\phi(\sigma) = (s(\sigma,o_1),\ldots,s(\sigma,o_k))$ where $\{o_1,\ldots,o_k\}$ is some fixed enumeration of $\calO$. Further, let $\psi(o_i) = e_i \in \mathbb{R}^k$ where the $i^{th}$ coordinate is $1$ and the rest are $0$. Then, for any profile $\pi$, we have
%\begin{align*}
%o_i \in r(\pi) &\Leftrightarrow \sum_{\sigma \in \rank} n(\pi,\sigma) \cdot s(\sigma,o_i) \ge \sum_{\sigma \in \rank} n(\pi,\sigma) \cdot s(\sigma,o), \forall o \in \calO\\
%&\Leftrightarrow \langle mean(\pi), e_i \rangle \ge \langle mean(\pi), e_j \rangle, \forall 1 \le j \le k \\
%&\Leftrightarrow \|mean(\pi) - e_i\| \ge \|mean(\pi) - e_j\|, \forall 1 \le j \le k\\
%&\Leftrightarrow \|mean(\pi) - \psi(o_i)\|^2 \ge \|mean(\pi) - \psi(o_j)\|^2, \forall 1 \le j \le k,\\
%&\Leftrightarrow \|mean(\pi) - \psi(o_i)\| \ge \|mean(\pi) - \psi(o_j)\|, \forall 1 \le j \le k,
%\end{align*}
%where the third transition follows since $\|e_j\| = 1$ for all $j$ (and thus, $\|mean(\pi) - \psi(o_j)\|^2 - \langle mean(\pi), e_j \rangle$ is constant for all $j$). Thus, $r$ is a mean proximity rule.
%\end{proof}
%\end{comment}

%%%%%%%%%%%%%%%%%%%%%%%%%%%%%%%%%%%%%%%%%%%%%

Due to this equivalence, any mean proximity rule has two representations: a pair of embeddings $(\psi,\phi)$, and a scoring function $s$, both of which may not be unique. Zwicker~\cite{Zwicker08b} also showed the following. %shows that the set of voting rules that are \emph{consistent} and \emph{connected} is identical to the set of mean neat voting rules, which is a generalization of mean proximity rules. As a simple corollary, we have: 

\cns{Check if continuity in~\cite{CRX09} is same as continuity in~\cite{Zwicker08a}. If so, take the definition from~\cite{Zwicker08a}, else from~\cite{CRX09}.}
\begin{proposition}[\cite{Zwicker08b}]
Any mean proximity rule is consistent, connected, continuous, and anonymous.
\label{prop:properties}
\end{proposition}

This implies that any voting rule that is not consistent (in the SWF sense) is not a mean proximity rule. In fact, we have the following.

\begin{lemma}[Proposition 1,2,5 and Theorem 3 of~\cite{CRX09}]
All positional scoring rules and the Kemeny rule are mean proximity rules. However, Bucklin's rule, Copeland's rule, the maximin rule, the ranked pairs method, and STV are not mean proximity rules since they do not satisfy consistency (under any tie-breaking scheme). 
\end{lemma}

%%%%%%%%%%%%%%%%%%%%%%%%%%%%%%%%%%%%%%%%%%%%%
%%%%%%%%%%%%%%%%%%%%%%%%%%%%%%%%%%%%%%%%%%%%%
%%%%%%%%%%%%%%%%%%%%%%%%%%%%%%%%%%%%%%%%%%%%%

\section{Symmetric Mean Proximity Rules}

In this paper, we are interested in social welfare functions (SWFs) that return a ranking (more formally, a set of tied rankings), so $\calO = \rank$. %From here onwards, unless mentioned otherwise, any voting rule we mention will be a social welfare function. 
Here, the scoring function $s : \rank \times \rank \rightarrow \mathbb{R}$ describes the \emph{similarity} between two rankings. This special case was also defined and studied by Conitzer et. al.~\cite{CRX09} as \emph{simple ranking scoring functions} (SRSFs). Under any fixed enumeration $\sigma_1,\ldots,\sigma_{m!}$, we can create an $m! \times m!$ matrix $S$ such that $S_{ij} = s(\sigma_i,\sigma_j)$. This is called a \emph{score matrix} of the generalized scoring rule (equivalently, mean proximity rule) represented by the scoring function $s$. 

%Further, given any profile $\pi$ of $n$ votes, we create the vector $y^{\pi}$ such that $y^{\pi}_i  = n(\pi,\sigma_i)/n$, for $1 \le i \le m!$. It is easy to verify that $\sigma_k$ would be in the output of the rule if the $k^{th}$ component of $S y^{\pi}$ is the maximum. 

%%%%%%%%%%%%%%%%%%%%%%%%%%%%%%%%%%%%%%%%%%%%%

While mean proximity rules capture several well-known SWFs, they capture bad voting rules as well. For example, we can imagine a mean proximity rule that has a score function $s$ satisfying $s(\sigma,\sigma') > s(\sigma,\sigma)$ for some $\sigma,\sigma' \in \rank$. In this case, the rule would not output $\sigma$ even on the profile where all votes are $\sigma$, thus violating unanimity. To solve this problem, we propose a simple fix. 

\begin{definition}[Symmetric Mean Proximity Rules]
A voting rule is called \emph{symmetric mean proximity rule} if there exists a mean proximity representation with identical input and output embeddings, i.e., $\psi = \phi$. 
\end{definition} 

Since the outcome space for SWFs is identical to the input space, $\psi = \phi$ is a natural restriction. It is easy to check that for any profile $\pi$ where all votes are $\sigma$, $mean(\pi) = \phi(\sigma)$. Thus, the rule would output the singleton set $\{\sigma\}$.\footnote{Rank-distinguishability plays a key role here. Any embedding of a rank-distinguishable symmetric mean proximity rule must map all rankings to different Euclidean points.} 

%%%%%%%%%%%%%%%%%%%%%%%%%%%%%%%%%%%%%%%%%%%%%

\begin{lemma}
While there exist mean proximity rules violating unanimity, all symmetric mean proximity rules satisfy unanimity.
\end{lemma}

Symmetric mean proximity rules are not too restrictive, as they still capture well-known mean proximity SWFs. 
\begin{lemma}
All positional scoring rules and the Kemeny rule are symmetric mean proximity rules.
\label{lem:symmetric-captures}
\end{lemma}

To see this, note that the constructions in~\cite{Zwicker08a} for positional scoring rules and the Kemeny rule actually use identical input and output embeddings. 
Recall that mean proximity rules are equivalent to generalized scoring rules (Proposition~\ref{prop:equiv}). It is natural to ask: \emph{What subclass of generalized scoring rules is equivalent to symmetric mean proximity rules?} 

%%%%%%%%%%%%%%%%%%%%%%%%%%%%%%%%%%%%%%%%%%%%%

\begin{definition}[Euclidean Distance Matrix (EDM)]
A $p \times p$ matrix $A = (a_{ij})$ is called a \emph{Euclidean distance matrix (EDM)} if there exist $v_1,\ldots,v_p \in \mathbb{R}^k$ such that $a_{ij} = \|v_i-v_j\|^2$, $\forall i,j$. 
\end{definition}

\begin{theorem}
A voting rule is a symmetric mean proximity rule if and only if it is a generalized scoring rule that has a score matrix whose negation is a Euclidean distance matrix. 
\label{thm:symm}
\end{theorem}
\begin{proof}
Note that Proposition~\ref{prop:MPR-GSR-conversion} shows the equivalence of a symmetric mean proximity rule with an embedding $\phi$ with the generalized scoring rule with a scoring function $s(\sigma,\sigma') = -\|\phi(\sigma)-\phi(\sigma')\|^2$. Note that the latter indeed generates a score matrix that is negation of an EDM.
\begin{comment} % Full Proof
For the ``if'' direction, given any generalized scoring rule $r$ with score matrix $S$ such that $-S$ is an EDM, we can find $v_1,\ldots,v_{m!} \in \mathbb{R}^k$ such that $S_{ij} = -\|v_i-v_j\|^2$. Take $\phi(\sigma_i) = v_i$ for all $i$. By Proposition~\ref{prop:MPR-GSR-conversion}, 
$$
\argmax_{\sigma \in \rank} \sum_{\sigma' \in \rank} n(\pi,\sigma') \cdot s(\sigma,\sigma') = \argmin_{\sigma \in \rank} \|\phi(\sigma)-mean(\pi)\|.
$$
That is, the rule is a symmetric mean proximity rule. For the ``only if'' direction, given any symmetric mean proximity rule, note that the score matrix created in the proof of Proposition~\ref{prop:equiv} is negation of a Euclidean distance matrix.
\end{comment}
\end{proof}

%%%%%%%%%%%%%%%%%%%%%%%%%%%%%%%%%%%%%%%%%%%%%

We showed that symmetric mean proximity rules improve upon mean proximity rules in three ways. First, taking identical input and output embeddings is very natural. Second, all symmetric mean proximity rules achieve unanimity. Third, the restriction of a mean proximity rule being symmetric is mild; all well-known mean proximity rules are symmetric. We also identified the subclass of generalized scoring rules that is equivalent to symmetric mean proximity rules. In the next section, we analyze symmetric mean proximity rules that satisfy another highly desired property -- neutrality. While neutrality is also mild (all voting rules of interest are neutral), we show that it adds significant structure to symmetric mean proximity rules. 

%%%%%%%%%%%%%%%%%%%%%%%%%%%%%%%%%%%%%%%%%%%%%
%%%%%%%%%%%%%%%%%%%%%%%%%%%%%%%%%%%%%%%%%%%%%
%%%%%%%%%%%%%%%%%%%%%%%%%%%%%%%%%%%%%%%%%%%%%

\section{Neutral Symmetric Mean Proximity Rules}

In this section, we focus on neutrality in symmetric mean proximity rules, and the restrictions it imposes on the scoring functions as well as the embeddings. First, we define neutral scoring functions as in~\cite{CRX09}, which are closely related to neutrality of generalized scoring rules. 

\cns{Consider switching from $\tau \sigma$ to $\tau(\sigma)$ to avoid formally dealing with interpretation of rankings as permutations. Or just say for a permutation $\tau$, the permuted ranking $\tau(\sigma)$ will be denoted by $\tau \sigma$ for notational convenience. Also take care of all $\tau \in \rank \rightarrow \tau \in S_m$.}

%%%%%%%%%%%%%%%%%%%%%%%%%%%%%%%%%%%%%%%%%%%%%

\begin{definition}[Neutral Scoring Function]
A scoring function $s: \rank \times \rank \rightarrow \mathbb{R}$ is called \emph{neutral} if $s(\tau \sigma, \tau \sigma') = s(\sigma,\sigma')$ for every $\sigma,\sigma',\tau \in \rank$. We say that a score matrix is neutral if the scoring function that generates the score matrix is neutral. 
\end{definition}
In words, a scoring function that describes similarity between two rankings is neutral if permuting two rankings in the same way does not change their similarity. Recall that Proposition~\ref{prop:MPR-GSR-conversion} links an embedding $\phi$ to the scoring function $s(\sigma,\sigma') = -\|\phi(\sigma)-\phi(\sigma')\|^2$. Thus, neutrality of a scoring function translates naturally to the following definition of neutrality on embeddings.

%%%%%%%%%%%%%%%%%%%%%%%%%%%%%%%%%%%%%%%%%%%%%

\begin{definition}[Neutral Embedding]
An embedding $\phi:\rank \rightarrow \mathbb{R}^k$ is called \emph{neutral} if $\|\phi(\sigma)-\phi(\sigma')\| = \|\phi(\tau \sigma)-\phi(\tau\sigma')\|$ for any $\tau,\sigma,\sigma' \in \rank$.
\end{definition}

%%%%%%%%%%%%%%%%%%%%%%%%%%%%%%%%%%%%%%%%%%%%%

Conitzer et. al.~\cite{CRX09} showed that neutrality of a generalized scoring rule (i.e., a mean proximity rule) can be translated to neutrality of its scoring function.
\begin{proposition}[Lemma 2, Conitzer et. al.~\cite{CRX09}]
A mean proximity rule is neutral if and only if it has a neutral scoring function.
\label{prop:gsr-neutral}
\end{proposition}

%%%%%%%%%%%%%%%%%%%%%%%%%%%%%%%%%%%%%%%%%%%%%

Proposition~\ref{prop:gsr-neutral} and Theorem~\ref{thm:symm} together allow us to identify neutral symmetric mean proximity rules as generalized scoring rules for which there exists a neutral scoring function as well as there exists a scoring function that generates a score matrix which is negation of an EDM. However, the two conditions cannot yet be combined. In the proof of Proposition~\ref{prop:gsr-neutral} given in~\cite{CRX09}, it is shown that if $s$ is any scoring function of a neutral GSR $r$, then $s^{\nt}$ defined by 
\begin{equation}
s^{\nt}(\sigma,\sigma') = \sum_{\tau \in \rank} s(\tau \sigma, \tau \sigma')
\label{eqn:s-nt}
\end{equation}
is neutral and represents the same voting rule $r$. We show that if we take the scoring function of the score matrix that is negation of an EDM and \emph{neutralize} it, then the resulting scoring function also generates a score matrix that is negation of an EDM. Further, let the original scoring function $s$ correspond to the embedding $\phi$, then we show that $s^{\nt}$ corresponds to the embedding $\phi^{\nt}$ given by 
\begin{equation}
\phi^{\nt}(\sigma) = [\phi(\tau_1 \sigma)^T \phi(\tau_2 \sigma)^T \ldots \phi(\tau_{m!} \sigma)^T]^T,
\label{eqn:phi-nt}
\end{equation}
where $\tau_1,\ldots,\tau_{m!}$ is any fixed enumeration of all permutations. Note that if $\phi$ is a $k$-dimensional embedding, then $\phi^{\nt}$ is an $m! \cdot k$-dimensional embedding. We consolidate these ideas into the following characterization.

%%%%%%%%%%%%%%%%%%%%%%%%%%%%%%%%%%%%%%%%%%%%%

\begin{theorem}
For any voting rule $r$, the following conditions are equivalent.
\begin{enumerate}
\item $r$ is a neutral symmetric mean proximity rule.
\item $r$ is a generalized scoring rule that has a score matrix which is neutral and negation of an EDM. 
\item $r$ is a symmetric mean proximity rule that has a neutral embedding.
\end{enumerate}
\label{thm:neutral-smpr}
\end{theorem}
\begin{proof}
First, it is easy to show that the second and the third conditions imply the first condition. If $r$ is a GSR that has a score matrix which is neutral and negation of an EDM, then by Proposition~\ref{prop:gsr-neutral} and Theorem~\ref{thm:symm}, $r$ is a neutral mean proximity rule and $r$ is a symmetric mean proximity rule, thus a neutral symmetric mean proximity rule. Also, if $r$ is a symmetric mean proximity rule with a neutral embedding $\phi$, then for any profile $\pi$, we have 
\begin{align*}
r(\tau \pi) &= \argmin_{\sigma \in \rank} \sum_{\sigma' \in \rank} n(\pi,\sigma') \cdot \|\phi(\sigma)-\phi(\tau \sigma')\|^2 \\
&= \tau \argmin_{\sigma \in \rank} \sum_{\sigma' \in \rank} n(\pi,\sigma') \cdot \|\phi(\tau \sigma)-\phi(\tau \sigma')\|^2 \\
&= \tau \argmin_{\sigma \in \rank} \sum_{\sigma' \in \rank} n(\pi,\sigma') \cdot \|\phi(\sigma)-\phi(\sigma')\|^2 \\
&= \tau r(\pi),
\end{align*}
where the first transition is due to Proposition~\ref{prop:MPR-GSR-conversion}, and the third transition is due to neutrality of $\phi$. Thus, $r$ is a neutral symmetric mean proximity rule. Now, we show that the first condition also implies the other two conditions.

Let $r$ be any neutral symmetric mean proximity rule. By Theorem~\ref{thm:symm}, $r$ has a score matrix $S$ (with scoring function $s$) that is negation of an EDM. Let $\phi$ be the embedding that generates $S$, i.e., $S_{ij} = -\|\phi(\sigma_i)-\phi(\sigma_j)\|^2$. Consider the embedding $\phi^{\nt}$ given by Equation~\eqref{eqn:phi-nt}. Also consider the corresponding scoring function $s^{\nt}$ given by Equation~\eqref{eqn:s-nt}. Let $S^{\nt}$ be the score matrix generated by the scoring function $s^{\nt}$. Then, 
\begin{align*}
S^{\nt}_{ij} &= s^{\nt}(\sigma_i,\sigma_j) = -\|\phi^{\nt}(\sigma_i)-\phi^{\nt}(\sigma_j)\|^2 \\
&= - \sum_{l=1}^{m!} \|\phi(\tau_l \sigma_i)-\phi(\tau_l \sigma_j)\|^2 = \sum_{\tau \in \rank} s(\tau \sigma_i, \tau \sigma_j).
\end{align*}
Conitzer et. al.~\cite{CRX09} showed (refer, proof of Lemma~2 in their paper) that the scoring function $s^{\nt}$ constructed this way corresponds to the same voting rule $r$ and $s^{\nt}$ is neutral. Thus, $S^{\nt}$ is a score matrix of $r$ that is neutral and negation of an EDM. Finally, it is easy to check that neutrality of $s^{\nt}$ implies neutrality of the embedding $\phi^{\nt}$ that we constructed.
\end{proof}

%%%%%%%%%%%%%%%%%%%%%%%%%%%%%%%%%%%%%%%%%%%%%
%%%%%%%%%%%%%%%%%%%%%%%%%%%%%%%%%%%%%%%%%%%%%
%%%%%%%%%%%%%%%%%%%%%%%%%%%%%%%%%%%%%%%%%%%%%

\section{Linear Embeddings}

Theorem~\ref{thm:neutral-smpr} allowed us to identify neutral symmetric mean proximity rules as those having a neutral embedding. %Checking neutrality of any given embedding is straightforward. 
Its constructive proof also gave an easy way of converting any embedding of a neutral symmetric mean proximity rule to an equivalent neutral embedding: just append embeddings of all permutations of the ranking in a fixed order. However, the dimension of the neutral embedding thus constructed is $m!$ times the dimension of the original embedding, which is huge. We are interested in \emph{generating low dimensional neutral embeddings}. Low dimensional embeddings are interesting because they are easy to work with and easy to visualize. 

\cns{cite more of relevant group theorey literature on linear representations and corresponding embeddings}

In this section, we introduce \emph{linear embeddings} by drawing ideas from representation theory of the symmetric group, where low dimensional representations have beem emphasized~\cite{Burnside12,KT12}. We then show that linear embeddings, while easy to construct, generate exactly the class of neutral symmetric mean proximity rules. 

%%%%%%%%%%%%%%%%%%%%%%%%%%%%%%%%%%%%%%%%%%%%%

\begin{definition}[Linear Embeddings]
An embedding $\phi:\rank \rightarrow \mathbb{R}^k$ is called \emph{linear} if there exists a function $R : \rank \rightarrow \mathbb{R}^{k \times k}$ mapping each permutation to a $k \times k$ real matrix such that i) $\phi(\tau \sigma) = R(\tau) \phi(\sigma)$ for every $\tau,\sigma \in \rank$, and ii) $R(\tau^{-1}) = R(\tau)^{-1} = R(\tau)^T$ for every $\tau \in \rank$. $R$ is called the representation of $\phi$. 
\end{definition}
Here, $\tau \sigma$ denotes the ranking obtained by permuting candidates in $\sigma$ according to $\tau$. In what follows, for notational convenience, we use $R_{\tau}$ instead of $R_{\tau}$. Note that linear embeddings are quite structured: embedding of a ranking $\sigma$ can be obtained from embedding of any other ranking $\sigma'$ by the linear transformation corresponding to the permutation that converts $\sigma'$ to $\sigma$. 

In the proof of Theorem~\ref{thm:neutral-smpr}, we took arbitrary embedding $\phi$ of a neutral symmetric mean proximity rule $r$, and constructed the neutral embedding $\phi^{\nt}$ given in Equation~\eqref{eqn:phi-nt}. We show that $\phi^{\nt}$ is also linear. Crucially, observe that for any $\sigma,\sigma' \in \rank$, the coordinates of $\phi^{\nt}(\sigma)$ and $\phi^{\nt}(\sigma')$ are just permutations of each other.
\begin{lemma}
For any embedding $\phi$, the neutral embedding $\phi^{\nt}$ given in Equation~\eqref{eqn:phi-nt} is linear.
\label{lem:nt-linear}
\end{lemma}
\begin{proof}
Let $\phi$ be a $k$-dimensional embedding. Hence, $\phi^{\nt}$ has dimension $m! \cdot k$. We need to show that there exists a representation $R$ such that for any $\tau,\sigma \in \rank$, $\phi^{\nt}(\tau \sigma) = R_{\tau}\phi^{\nt}(\sigma)$. Note that 
$$
\phi^{\nt}(\tau \sigma) = [\phi(\tau_1 \tau \sigma)^T \phi(\tau_2 \tau \sigma)^T \ldots \phi(\tau_{m!} \tau \sigma)^T]^T.
$$
Let $\Pi_{\tau}$ be the $m! \times m!$ matrix such that $\Pi_{ij} = 1$ if and only if $\tau_j = \tau_i \tau$. It is easy to verify that $\Pi_{\tau}$ is a permutation matrix. Further, if the blocks of $\phi^{\nt}(\sigma)$ were permuted according to $\Pi_{\tau}$, then the $i^{th}$ block in the resulting vector will be $\phi(\tau_j \sigma)$ such that $\tau_j = \tau_i \tau$. That is, the $i^{th}$ block in the resulting vector will be $\phi(\tau_i \tau \sigma)$, i.e., the $i^{th}$ block of $\phi^{\nt}(\tau \sigma)$. Hence, applying $\Pi_{\tau}$ to the blocks of $\phi^{\nt}(\sigma)$ results in $\phi^{\nt}(\tau \sigma)$. Now, it is easy to see that if we construct an $m! \cdot k \times m! \cdot k$ matrix $R_{\tau}$ by replacing every $1$ in $\Pi_{\tau}$ by a $k\times k$ identity matrix, then $R_{\tau} \phi^{\nt}(\sigma) = \phi^{\nt}(\tau \sigma)$. Finally, it is also easy to verify that $R_{\tau^{-1}} = R_{\tau}^{-1} = R_{\tau}^T$ is satisfied for all $\tau$, because $\Pi_{\tau^{-1}} = \Pi_{\tau}^{-1} = \Pi_{\tau}^T$ is satisfied by the corresponding $\Pi_{\tau}$ that they are generated from.
\end{proof}

Combining the proof of Theorem~\ref{thm:neutral-smpr} with Lemma~\ref{lem:nt-linear}, we know that every neutral symmetric mean proximity rule has a linear embedding. However, this is still the high-dimensional embedding $\phi^{\nt}$ we constructed earlier. We would like to know if other low dimensional linear embeddings are also interesting. We also show another interesting property that all linear embeddings satisfy. 

\begin{definition}[Equal Norm Embedding]
An embedding $\phi : \rank \rightarrow \mathbb{R}^k$ is said to have \emph{equal norm} if $\|\phi(\sigma)\| = \|\phi(\sigma')\|$ for all $\sigma,\sigma' \in \rank$.
\end{definition}

\begin{lemma}
Any linear embedding is an equal norm neutral embedding, and hence represents a neutral symmetric mean proximity rule.
\label{lem:linear-neutral}
\end{lemma}
\begin{proof}
Let $\phi$ be any linear embedding and let $R$ be its representation. Then for any $\tau,\sigma,\sigma' \in \rank$,
$$
\langle \phi(\tau \sigma), \phi(\tau \sigma') \rangle = \langle R_{\tau}\phi(\sigma), R_{\tau}\phi(\sigma') \rangle = \langle \phi(\sigma), R_{\tau}^T R_{\tau}\phi(\sigma') \rangle = \langle \phi(\sigma), \phi(\sigma') \rangle.
$$
Here, the last transition holds because $R_{\tau}$ is an orthogonal matrix by definition. Taking $\sigma = \sigma'$ yields that $\phi$ is an equal norm embedding. Now for any $\tau,\sigma,\sigma' \in \rank$,
\begin{align*}
\|\phi(\tau \sigma)-\phi(\tau \sigma')\|^2 &= \|\phi(\tau \sigma)\|^2 + \|\phi(\tau \sigma')\|^2 - 2\cdot \langle \phi(\tau \sigma), \phi(\tau \sigma') \rangle \\
&= \|\phi(\sigma)\|^2 + \|\phi(\sigma')\|^2 - 2\cdot \langle \phi(\sigma), \phi(\sigma') \rangle = \|\phi(\sigma)-\phi(\sigma')\|^2.
\end{align*}
Thus, $\phi$ is neutral. By Theorem~\ref{thm:neutral-smpr}, it represents a neutral symmetric mean proximity rule.
\end{proof}

Combining Lemmas~\ref{lem:nt-linear} and~\ref{lem:linear-neutral} gives the following characterization.
\begin{theorem}
A symmetric mean proximity rule is neutral if and only if it has a linear embedding.
\label{thm:linear-char}
\end{theorem}

\subsection{Connection between Neutrality and Linearity}

We saw that every linear embedding is a neutral embedding. In the other direction, every neutral embedding may not be linear. For example, a neutral embedding may not even be an equal norm embedding, while every linear embedding is an equal norm embedding. We showed that the particular neutral embedding $\phi^{\nt}$ is indeed linear, but it has very high dimension. \emph{How do low dimensional neutral embeddings connect with linear embeddings?} We show that for every neutral embedding, we can very easily construct an equivalent, equal or lower dimensional, equal norm neutral embedding that is also linear. For this, we first need some important results about embeddings of neutral symmetric mean proximity rules.

\begin{lemma}
Let $r$ be any neutral voting rule. Let $\pi_{symm}$ be the profile containing each ranking exactly once, i.e., $n(\pi_{symm},\sigma) = 1$ for all $\sigma \in \rank$. Then, $r(\pi_{symm}) = \rank$. 
\label{lem:average-profile}
\end{lemma}
\begin{proof}
Let $r(\pi_{symm}) = T \subseteq \rank$. Suppose $T \neq \rank$, so there exists a $\sigma' \notin T$. Further, by definition of a voting rule, $T \neq \emptyset$. Thus, there exists a $\sigma \in T$. Now, take $\tau$ to be the permutation that sends $\sigma$ to $\sigma'$. % = \sigma' \sigma^{-1}$, where $\sigma^{-1}$ is the inverse of $\sigma$ in the symmetric group $S_m$. 
It is easy to see that $\tau \pi_{symm} = \pi_{symm}$. Hence, $r(\tau \pi_{symm}) = r(\pi_{symm}) = T$. Thus, $\sigma' \notin r(\tau \pi_{symm})$. However, $\sigma' \in \tau r(\pi_{symm})$. Thus, $\tau r(\pi_{symm}) \neq r(\tau \pi_{symm})$. This implies that $r$ violates neutrality, a contradiction. 
\end{proof}

Therefore, for any embedding $\phi$ of a neutral symmetric mean proximity rule, $\|\phi(\sigma)-mean(\pi_{symm})\|$ must be a constant independent of $\sigma$ so that all rankings are tied on $\pi_{symm}$. Let $\phi_{avg} = (1/{m!}) \cdot \sum_{\sigma \in \rank} \phi(\sigma)$. Define an embedding $\hat{\phi}$  such that $\hat{\phi}(\sigma) = \phi(\sigma) - \phi_{avg}$ for all $\sigma \in \rank$. We call $\hat{\phi}$ the \emph{normalization} of $\phi$. We argued that it is an equal norm embedding. Note that normalization of an embedding is just a translation. Intuitively, this does not change the geometry of the points to which the rankings are embedded, and hence should represent the same symmetric mean proximity rule. We show that in addition, normalization also preserves neutrality and linearity. 

%%%%%%%%%%%%%%%%%%%%%%%%%%%%%%%%%%%%%%%%%%%%%

\begin{lemma}
Let $\phi$ be any embedding of a neutral symmetric mean proximity rule $r$. Then, its normalization $\hat{\phi}$ is an equal norm embedding that also represents $r$. Further, normalization preserves both neutrality and linearity.\footnote{Here, preservation of a property means that if the former embedding satisfies it, then the latter embedding does too.}
\label{lem:preservation}
\end{lemma}
\begin{proof}
Let $\phi : \rank \rightarrow \mathbb{R}^k$ be any embedding of a neutral symmetric mean proximity rule $r$. Consider its normalization $\hat{\phi}$. From Lemma~\ref{lem:average-profile}, it is clear that all $\phi(\sigma)$ must be at equal distance from $mean(\pi_{symm}) = \phi_{avg}$. Hence, $\hat{\phi} = \phi-\phi_{avg}$ must be an equal norm embedding. Further, for any rankings $\sigma, \sigma' \in \rank$, $\|\hat{\phi}(\sigma)-\hat{\phi}(\sigma')\| = \|\phi(\sigma)-\phi_{avg}-\phi(\sigma')+\phi_{avg}\| = \|\phi(\sigma)-\phi(\sigma')\|$. Now by Proposition~\ref{prop:MPR-GSR-conversion}, for any profile $\pi$,
\begin{align*}
r(\pi) &= \argmin_{\sigma \in \rank} \sum_{\sigma' \in \rank} n(\pi,\sigma') \cdot \|\phi(\sigma)-\phi(\sigma')\|^2 \\
&= \argmin_{\sigma \in \rank} \sum_{\sigma' \in \rank} n(\pi,\sigma') \cdot \|\hat{\phi}(\sigma)-\hat{\phi}(\sigma')\|^2.
\end{align*}
Thus, $\hat{\phi}$ also represents $r$. 

If $\phi$ is neutral, then for any $\tau,\sigma,\sigma' \in \rank$, we have $\|\hat{\phi}(\tau \sigma)-\hat{\phi}(\tau \sigma')\| = \|\phi(\tau \sigma)-\phi(\tau \sigma')\| = \|\phi(\sigma)-\phi(\sigma')\| = \|\hat{\phi}(\sigma)-\hat{\phi}(\sigma')\|$. Hence, $\hat{\phi}$ is neutral too. 

Finally, assume $\phi$ is linear with representation $R$. First, we have that for any $\tau,\sigma \in \rank$, $\phi(\tau \sigma) = R_{\tau}\phi(\sigma)$. Averaging over all $\sigma$, we get that 
$$
\frac{1}{m!} \sum_{\sigma \in \rank} \phi(\tau \sigma) = \frac{1}{m!}  \sum_{\sigma \in \rank} R_{\tau} \phi(\sigma) = R_{\tau} \left (\frac{1}{m!}  \sum_{\sigma \in \rank} \phi(\sigma) \right).
$$
However, 
$$
\frac{1}{m!}  \sum_{\sigma \in \rank} \phi(\tau \sigma) = \frac{1}{m!}  \sum_{\sigma \in \rank} \phi(\sigma) = \phi_{avg}.
$$
Hence, we have $R_{\tau}\phi_{avg} = \phi_{avg}$ for all $\tau \in \rank$. Now, for any $\tau,\sigma \in \rank$, 
$$
\hat{\phi}(\tau \sigma) = \phi(\tau \sigma) - \phi_{avg} = R_{\tau}\phi(\sigma) - R_{\tau}\phi_{avg} = R_{\tau}\hat{\phi}(\sigma).
$$
Thus, $\hat{\phi}$ is also linear with the same representation $R$.
%Let $\phi'$ be arbitrary embedding generated by translating and scaling $\phi$, that is, for some $a \in \mathbb{R}\setminus\{0\}$ and $b \in \mathbb{R}^k$, $\phi'(\sigma) = a \cdot \phi(\sigma) + b$ for all $\sigma \in \rank$. For any rankings $\sigma$ and $\sigma'$, $\|\phi'(\sigma)-\phi'(\sigma')\| = \|a\phi(\sigma)+b-a\phi(\sigma')-b\| = |a| \cdot \|\phi(\sigma)-\phi(\sigma')\|$. Now by Proposition~\ref{prop:MPR-GSR-conversion}, for any profile $\pi$,
%\begin{align*}
%r(\pi) &= \argmin_{\sigma \in \rank} \sum_{\sigma' \in \rank} n(\pi,\sigma') \cdot \|\phi(\sigma)-\phi(\sigma')\|^2 \\
%&= \argmin_{\sigma \in \rank} a^2 \cdot \sum_{\sigma' \in \rank} n(\pi,\sigma') \cdot \|\phi(\sigma)-\phi(\sigma')\|^2 \\
%&= \argmin_{\sigma \in \rank} \sum_{\sigma' \in \rank} n(\pi,\sigma') \cdot \|\phi'(\sigma)-\phi'(\sigma')\|^2,
%\end{align*}
%where the second transition follows since $a^2 > 0$. This shows that $\phi'$ also represents $r$. 
%
%Further, if $\phi$ is neutral, then for any $\tau,\sigma,\sigma' \in \rank$, we have $\|\phi'(\tau \sigma)-\phi'(\tau \sigma')\| = a \cdot \|\phi(\tau \sigma)-\phi(\tau \sigma')\| = a \cdot \|\phi(\sigma)-\phi(\sigma')\| = \|\phi'(\sigma)-\phi'(\sigma')\|$. Hence, $\phi'$ is neutral as well. 
%
%Finally, assume $\phi$ is linear with representation $R$ and $\phi'$ is the normalization of $\phi$. First, we have that for any $\tau,\sigma \in \rank$, $\phi(\tau \sigma) = R_{\tau}\phi(\sigma)$. Averaging over all $\sigma$, we get that $(1/{m!}) \sum_{\sigma \in \rank} \phi(\tau \sigma) = (1/{m!}) \sum_{\sigma \in \rank} R_{\tau} \phi(\sigma)$. However, $(1/{m!}) \sum_{\sigma \in \rank} \phi(\tau \sigma) = (1/{m!}) \sum_{\sigma \in \rank} \phi(\sigma) = \phi_{avg}$. Hence, we have $R_{\tau}\phi_{avg} = \phi_{avg}$ for all $\tau \in \rank$. Let $c = \|\phi(\sigma)-\phi_{avg}\|$ (equal for all $\sigma \in \rank$). Then, for any $\tau,\sigma \in \rank$, 
%$$
%\phi'(\tau \sigma) = (1/c) \cdot (\phi(\tau \sigma) - \phi_{avg}) = (1/c) \cdot (R_{\tau}\phi(\sigma) - R_{\tau}\phi_{avg}) = R_{\tau}\phi'(\sigma).
%$$
%Thus, $\phi'$ is also linear with the same representation $R$.
\end{proof}

Thus, given any neutral embedding of a neutral symmetric mean proximity rule $r$, Lemma~\ref{lem:preservation} says that its normalization $\hat{\phi}$ is an equal norm neutral embedding representing $r$. Note that equal norm is a necessity to achieve linearity since every linear embedding has equal norm. While $\hat{\phi}$ may not itself be linear, we show that we can construct an equal or lower dimensional linear embedding from it without losing neutrality. \cns{Sebastien, I believe all equal norm neutral embeddings are linear. The following proof is very close to showing that. It shows existence of $R_{\tau}$'s. Just need to show existence of a solution that satisfies $R_{\tau^{-1}} = R_{\tau}^{-1} = R_{\tau}^T$. If we can show that, we can put it as a lemma and say that $\hat{\phi}$ is linear. That would be awesome.} 

\begin{theorem}
For any neutral embedding, there exists an equivalent embedding of equal or lower dimension that is both neutral and linear. 
\label{thm:neutral-linear}
\end{theorem}

\cns{OVERVIEW}

Neutrality results (equal distance from avg), translation preserves everything.

Big Theorem:
Neutral -> subtract avg = equal norm neutral and forms subspace passing through origin -> take basis coefficients = smaller dimensional full rank representation -> Full rank equal norm neutral = linear. 


\cns{OVERVIEW}



Consider any neutral embedding $\phi: \rank \rightarrow \mathbb{R}^k$. Let $k' \le k$ be the dimension of the affine span of the points $\phi(\sigma_1),\ldots,\phi(\sigma_{m!})$. If $k' \neq k$, then we can transform the points to $\mathbb{R}^{k'}$ \cns{Sebastien, I need your help to formally do this. I know this is easy, just can't seem to get a handle on it.} to get another embedding $\phi'$ that preserves all distances, and thus represents the same symmetric mean proximity rule. Further, $\phi'$ is $k'$-dimensional and $\phi'(\sigma_1),\ldots,\phi'(\sigma_{m!})$ span $\mathbb{R}^{k'}$. This is formalized below.

\begin{definition}[Minimal Dimension Embedding]
An embedding $\phi: \rank \rightarrow \mathbb{R}^k$ is said to have \emph{minimal dimension} if the affine span of $\phi(\sigma_1),\ldots,\phi(\sigma_{m!})$ is $\mathbb{R}^k$.
\end{definition}
Later in Section~\ref{sec:min-dim}, we explore the minimum dimension required to embed any symmetric mean proximity rule. The embeddings which achieve this dimension are called \emph{minimum dimension} embeddings. It should be noted that not every minimal dimension embedding may have the minimum dimension. 


Now, we prove the main result of this section.
\begin{theorem}
Every minimal dimension neutral embedding is linear.
\label{thm:neutral-linear}
\end{theorem}
\begin{proof}
Let $\phi$ be any minimal dimension equal norm neutral embedding. We want to show that $\phi$ is linear. Assume $c = \|\phi(\sigma)\|$ for every $\sigma \in \rank$. Now, for any $\tau,\sigma,\sigma' \in \rank$,
\begin{align*}
\langle \phi(\tau \sigma), \phi(\tau \sigma') \rangle &= \frac{1}{2} \cdot \left(2\cdot c^2 - \|\phi(\tau \sigma) - \phi(\tau \sigma')\|^2\right) \\
&= \frac{1}{2} \cdot \left(2\cdot c^2 - \|\phi(\sigma) - \phi(\sigma')\|^2\right) = \langle \phi(\sigma), \phi(\sigma') \rangle.
\end{align*}
Take $A = [\phi(\sigma_1) \ldots \phi(\sigma_{m!}]$ (columns are $\phi(\sigma_i)$) and $B = [\phi(\tau \sigma_1) \ldots \phi(\tau \sigma_{m!})]$ (columns are $\phi(\tau \sigma_i)$). Then, $(B^T B)_{ij} = \langle \phi(\tau \sigma), \phi(\tau \sigma') \rangle = \langle \phi(\sigma), \phi(\sigma') \rangle = (A^T A)_{ij}$ for all $i,j$, i.e., $A^T A = B^T B$. 

We now show that for all $\tau \in \rank$, there exists a matrix $R_{\tau}$ such that $\phi(\tau \sigma) = R_{\tau} \phi(\sigma)$ for all $\sigma \in \rank$. Combining all the equations, this is equivalent to existence of a solution of $R_{\tau} A = B$, i.e., existence of an $X$ such that $A^T X = B^T$. It is well-known that the system of equations $Ax=b$ has a solution if and only if $AA^{+}b = b$, where $A^{+}$ is the Moore-Penrose pseudoinverse of $A$. Trivially extending this to systems of linear equations, we can see that the necessary and sufficient condition for existence of the required $R_{\tau}$ is $A^T (A^T)^{+} B^T = B^T$, or equivalently, $B A^{+} A = B$. The last derivation uses the fact that $(A^T)^{+} = (A^{+})^T$. However, note that
\begin{align*}
B &= B B^{+} B = B \left( (B^T B)^{+} B^T \right) B = B (B^T B)^{+} \left( B^T B \right) \\
&= B (A^T A)^{+} \left( A^T A \right) = B \left( (A^T A)^{+} A^T \right) A = B A^{+} A.
\end{align*}
Refer~\cite{BH12} for the identities $X = X X^{+} X$ (used in the first transition) and $X^{+} = (X^T X)^{+} X^T$ (used in the second and the fifth transitions) regarding Moore-Penrose pseudoinverses. The fourth transition follows since $A^T A = B^T B$. Therefore, the required $R_{\tau}$ exists. Next, we show that $R_{\tau^{-1}} = R_{\tau}^{-1} = R_{\tau}^T$. 
\end{proof}

\cns{POLISHED TILL HERE}\\
\cns{POLISHED TILL HERE}\\
\cns{POLISHED TILL HERE}

%%%%%%%%%%%%%%%%%%%%%%%%%%%%%%%%%%%%%%%%%%%%%

We have shown that every linear embedding is an equal norm embedding. For linear embeddings to generate all neutral symmetric mean proximity rule, it is essential for every neutral symmetric mean proximity rule to have an equal norm embedding in the first place. Our next aim is to show this. 

%%%%%%%%%%%%%%%%%%%%%%%%%%%%%%%%%%%%%%%%%%%%%

%%%%%%%%%%%%%%%%%%%%%%%%%%%%%%%%%%%%%%%%%%%%%

With the help of Lemma~\ref{lem:preservation}, we can show the following characterization, which is at the core of our main result of this section.
\begin{theorem}
An embedding is an equal norm neutral embedding if and only if it is a linear embedding.
\label{thm:neutral-linear}
\end{theorem}
\begin{proof}
The ``if'' direction is really simple. Let $\phi$ be any linear embedding with representation $R$. By Lemma~\ref{lem:linear-neutral}, $\phi$ is neutral as well. 


\cns{If this way does not work, then directly mention that above constructed embedding is linear because a permutation representation works for it.}

\end{proof}

\begin{theorem}
\end{theorem}

\cns{OVERVIEW OF THE REST}

We also need the following result, which allows converting the square of Euclidean distance in Proposition~\ref{prop:MPR-GSR-conversion} to inner products for for equal norm embeddings.
\begin{lemma}
For any equal norm embedding $\phi$ of a symmetric mean proximity rule $r$, $s(\sigma,\sigma') = \langle \phi(\sigma),\phi(\sigma') \rangle$ is a scoring function of $r$.
\label{lem:inner-product}
\end{lemma}
\begin{proof}
Let $c = \|\phi(\sigma)\|$, which is independent of $\sigma$ since $\phi$ is an equal norm embedding. From Proposition~\ref{prop:MPR-GSR-conversion}, we know that on any profile $\pi$, 
\begin{align*}
r(\pi) &= \argmin_{\sigma \in \rank} \sum_{\sigma' \in \rank} n(\pi,\sigma') \cdot \|\phi(\sigma)-\phi(\tau \sigma')\|^2 \\
&= \argmin_{\sigma \in \rank} \sum_{\sigma' \in \rank} n(\pi,\sigma') \cdot \left( c^2 + c^2 - 2 \cdot \langle \phi(\sigma), \phi(\sigma')\rangle\right) \\
&= \argmin_{\sigma \in \rank} 2 \cdot n \cdot c^2 - 2 \cdot \sum_{\sigma' \in \rank} n(\pi,\sigma') \langle \phi(\sigma), \phi(\sigma')\rangle \\
&= \argmax_{\sigma \in \rank} \sum_{\sigma' \in \rank} n(\pi,\sigma') \langle \phi(\sigma), \phi(\sigma')\rangle.
\end{align*}
It is now clear that $s(\sigma,\sigma') = \langle \phi(\sigma),\phi(\sigma') \rangle$ is indeed a scoring function of $r$.
\end{proof}

%%%%%%%%%%%%%%%%%%%%%%%%%%%%%%%%%%%%%%%%%%%%%

Above in Vince's theorem, say what the new function s' is. Introduce neutral embeddings as the only way to combine them. Mention why combining is not easy, since the s' may not preserve negation of EDM. Then motivate the whole equalizing norm => equal norm -EDM preserved by s' but not every neutral -EDM is neutral SMPR. Equivalently PSD score matrix - also preserved by s', equal neutral embeddings equivalent to linear embeddings. Both these are sufficient as well. Final theorem:
Neutral SMPR iff
1) Equal neutral, i.e., linear embedding
2) GSR with neutral, PSD score matrix.

Also use this to analyze all 3-alternative 2-dimensional embedding neutral SMPR. => R(id) = I, R(clockwise) and R(counterclockwise) = 120 deg add and subtract, R(exchange 12) = flip around theta, R(exchange 23) = R(exchange 12)*R(clockwise), R(exchange 13) = R(exchange 12)*R(counterclockwise). Thus, only parameter is theta. All theta work.

\cns{OVERVIEW COMPLETE}


Further, it is trivial to observe that translating and scaling all embeddings equally in the Euclidean space does not change the output of a symmetric mean proximity rule. Thus:
\begin{lemma}
Let $\phi$ be any embedding of a neutral symmetric mean proximity rule $r$. Then, the embedding $\phi'$, called \emph{normalized embedding}, given by
$$
\phi'(\sigma) = \frac{\phi(\sigma)-\phi_{avg}}{\|\phi(\sigma)-\phi_{avg}\|}
$$
represents the same voting rule $r$. Further, it satisfies $\|\phi'(\sigma)\| = 1$ for all $\sigma \in \rank$, and $\phi'_{avg} = 0$.
\label{lem:normalized-embedding}
\end{lemma}

%%%%%%%%%%%%%%%%%%%%%%%%%%%%%%%%%%%%%%%%%%%%%
%%%%%%%%%%%%%%%%%%%%%%%%%%%%%%%%%%%%%%%%%%%%%

Given any symmetric mean proximity rule with embedding $\phi$, Theorem~\ref{thm:symm} constructs a score matrix $S$ such that $S_{ij} = -\|\phi(\sigma_i)-\phi(\sigma_j)\|^2$, which is negated EDM. For normalized embeddings, we can find a more structured score matrix.

\begin{definition}[Gramian Matrix]
A $p \times p$ matrix $A = (a_{ij})$ is called \emph{Gramian} if there exist vectors $v_1,\ldots,v_p \in \mathbb{R}^k$ such that $a_{ij} = \langle v_i,v_j \rangle$ for all $i,j$. It is well-known that a matrix is Gramian if and only if it is positive semidefinite. 
\end{definition}

\begin{lemma}
Let $\phi$ be any embedding of a symmetric mean proximity rule $r$ such that $\|\phi(\sigma)\|$ is independent of $\sigma$. Then, the scoring function $s(\sigma,\sigma') = \langle \phi(\sigma),\phi(\sigma') \rangle$ represents the same rule $r$. The score matrix of this scoring function is Gramian, and hence positive semidefinite.
\label{lem:score-inner-products}
\end{lemma}
\begin{proof}
Let $\|\phi(\sigma)\| = c$ for all $\sigma \in \rank$. We have
\begin{align*}
&\argmin_{\sigma \in \rank} \|\phi(\sigma)-mean(\pi)\| \\
&= \argmin_{\sigma \in \rank} \sum_{\sigma' \in \rank} n(\pi,\sigma') \cdot \|\phi(\sigma)-\phi(\sigma')\|^2 \\
&= \argmin_{\sigma \in \rank} \sum_{\sigma' \in \rank} n(\pi,\sigma') \cdot \left( c^2 + c^2 - 2 \cdot \langle \phi(\sigma), \phi(\sigma') \rangle \right)\\
&= \argmax_{\sigma \in \rank} \sum_{\sigma' \in \rank} n(\pi,\sigma') \cdot \langle \phi(\sigma),\phi(\sigma') \rangle,
\end{align*}
where the first transition follows from Proposition~\ref{prop:MPR-GSR-conversion}. By definition, the score matrix of this scoring function is Gramian, and hence positive semidefinite.
\end{proof}

%%%%%%%%%%%%%%%%%%%%%%%%%%%%%%%%%%%%%%%%%%%%%


\begin{theorem}
A voting rule is a neutral symmetric mean proximity rule if and only if it is a neutral generalized scoring rule for which there exists a positive semidefinite score matrix with equal diagonal entries. 
\label{thm:psd}
\end{theorem}
\begin{proof}
Let $r$ be any neutral symmetric mean proximity rule. Then, by Lemma~\ref{lem:normalized-embedding}, it has a normalized embedding $\phi$. Recall that normalized embeddings satisfy $\langle \phi(\sigma),\phi(\sigma) \rangle = 1$ for all $\sigma \in \rank$. Thus, by Proposition~\ref{prop:equiv-dist-inner-product}, for any profile $\pi$,
$$
r(\pi) = \argmin_{\sigma \in \rank} \|\phi(\sigma)-mean(\pi)\| = \argmax_{\sigma \in \rank} \sum_{\sigma' \in \rank} n(\pi,\sigma') \cdot \langle \phi(\sigma),\phi(\sigma') \rangle.
$$
It is now evident that the scoring function $s(\sigma,\sigma') = \langle \phi(\sigma),\phi(\sigma') \rangle$ represents rule $r$. Further, it corresponds to a Gramian score matrix, which is positive semidefinite (Proposition~\ref{prop:gramian}). Finally, the diagonal entries $\langle \phi(\sigma),\phi(\sigma) \rangle$ are equal to $1$ since $\phi$ is a normalized embedding.

On the contrary, take any neutral generalized scoring rule $r$ with a positive semidefinite (and hence Gramian) score matrix $S$ and equal diagonal entries. Thus, there exist vectors $v_1,\ldots,v_{m!} \in \mathbb{R}^k$ such that $S_{ij} = \langle v_i,v_j \rangle$. Take $\phi(\sigma_i) = v_i$ for all $i$. Since the diagonal entries are equal, we have that $\langle \phi(\sigma),\phi(\sigma) \rangle$ is independent of $\sigma$. Thus, Proposition~\ref{prop:equiv-dist-inner-product} implies that on any profile $\pi$, the rule returns 
\begin{align*}
r(\pi) = \argmax_{\sigma_i \in \rank} \sum_{\sigma_j \in \rank} n(\pi,\sigma_j) \cdot S_{ij} &= \argmax_{\sigma \in \rank} \sum_{\sigma' \in \rank} n(\pi,\sigma') \cdot \langle \phi(\sigma),\phi(\sigma') \rangle \\
&= \argmin_{\sigma \in \rank} \|\phi(\sigma)-mean(\pi)\|.
\end{align*}
This shows that $r$ is a symmetric mean proximity rule with embedding $\phi$. Note that $r$ is already neutral by assumption. 
\end{proof}

\cns{Ideal Goal: Find a score matrix satisfying both 1) and 2). Note that the score matrix of a linear embedding satisfies both. Improve 3) to be the corresponding neutral + normalized embedding.}

We now prove several properties of representations of a neutral symmetric mean proximity rule. 
\begin{lemma}
Any neutral symmetric mean proximity rule $r$ has the following special representations. 
\begin{enumerate}
\item A score matrix $S$ satisfying the following.
\begin{enumerate}
\item $S$ is positive semidefinite (hence, symmetric).
\item $S_{ii} = S_{jj}$ for all $i,j$, i.e., all diagonal entries of $S$ are equal.
\item $S_{ii} > S_{ij} = S_{ji}$ for all $i,j$, i.e., the largest element in any row or column is its diagonal element.
\item $\sum_j S_{ij} = \sum_{j} S_{ji} = 0$ for all $i$, i.e., the sum across each row and column is $0$. 
\end{enumerate}
\item A score matrix $S$ generated by a neutral scoring function $s$, i.e., $S_{ij} = s(\sigma_i,\sigma_j)$ for all $i,j$ and $s(\tau \sigma,\tau \sigma') = s(\sigma,\sigma')$ for all $\tau,\sigma,\sigma' \in \rank$.
\item A normalized embedding $\phi$ that satisfies $\|\phi(sigma)\| = 1$ for all $\sigma \in \rank$, and $\sum_{\sigma \in \rank} \phi(\sigma) = 0$.
\end{enumerate}
\label{lem:neutral-mean-proximity-properties}
\end{lemma}
\begin{proof}
For the first representation, take the score matrix $S$ obtained by inner products of a normalized embedding $\phi$ as in the proof of Theorem~\ref{thm:psd}. We already know that $S$ is positive semi-definite with equal diagonal entries. Condition (c) follows by unanimity of symmetric mean proximity rules (note that $S$ is symmetric, so $S_{ij} = S_{ji}$). Condition (d) follows since the sum across row (column) $i$ is $\langle \phi(\sigma_i),\sum_{\sigma \in \rank} \phi(\sigma) \rangle = 0$ as $\sum_{\sigma \in \rank} \phi(\sigma) = 0$ for any normalized embedding $\phi$. 

The second representation follows by Proposition~\ref{prop:gsr-neutral}, and the third representation follows by Lemma~\ref{lem:normalized-embedding}. 
\end{proof}


In the next section, we introduce a general class of embeddings such that all of them satisfy neutrality, and use Lemma~\ref{lem:neutral-mean-proximity-properties} to show that they generate all neutral symmetric mean proximity rules.

%%%%%%%%%%%%%%%%%%%%%%%%%%%%%%%%%%%%%%%%%%%%%
%%%%%%%%%%%%%%%%%%%%%%%%%%%%%%%%%%%%%%%%%%%%%
%%%%%%%%%%%%%%%%%%%%%%%%%%%%%%%%%%%%%%%%%%%%%

First, we show that linearity is not a restrictive condition for symmetric mean proximity rules.

\begin{lemma}
All positional scoring rules and the Kemeny rule are linear symmetric mean proximity rules.
\label{lem:linear-captures}
\end{lemma}
\begin{proof}
\cns{To be written}
\end{proof}

Next, we prove one of the main results of this paper that shows the importance of linear mean proximity rules. 
\begin{theorem}
A mean proximity rule is neutral if and only if it is a linear mean proximity rule. 
\label{thm:linear-neutral}
\end{theorem}
\begin{proof}
\cns{One direction to be written, another to be proved}
\end{proof}

%%%%%%%%%%%%%%%%%%%%%%%%%%%%%%%%%%%%%%%%%%%%%
%%%%%%%%%%%%%%%%%%%%%%%%%%%%%%%%%%%%%%%%%%%%%
%%%%%%%%%%%%%%%%%%%%%%%%%%%%%%%%%%%%%%%%%%%%%

%%%%%%%%%%%%%%%%%%%%%%%%%%%%%%%%%%%%%%%%%%%%%
%%%%%%%%%%%%      UNPOLISHED SECTION       %%%%%%%%%%%%%%%%%%
%%%%%%%%%%%%%%%%%%%%%%%%%%%%%%%%%%%%%%%%%%%%%

\section{Connections to other approaches}
Voting rules have been analyzed from three viewpoints in the literature \cns{cite}: axiomatic view, distance rationalizability (DR) view, and MLE view. We connect linear mean proximity rules with all three viewpoints.

%%%%%%%%%%%%%%%%%%%%%%%%%%%%%%%%%%%%%%%%%%%%%

\subsection{Axiomatic View}
Proposition~\ref{prop:properties} shows that all mean proximity rules are consistent and connected. Theorem~\ref{thm:linear-neutral} shows that linear mean proximity rules are also neutral. This raises a few questions.

\begin{enumerate}
\item What restrictions on the embedding $\phi$ correspond to monotonicity of the mean proximity rule?
\item Caragiannis et. al.~\cite{CPS13} introduced pairwise majority consistency (PM-c) as the natural generalization of Condorcet consistency for social welfare functions. What restrictions on $\phi$ ensures that the mean proximity rule is PM-c? Is that related to the distance function $d(\sigma,\sigma') = \|\phi(\sigma)-\phi(\sigma')\|$ being majority concentric~\cite{CPS13}?
\end{enumerate}

%%%%%%%%%%%%%%%%%%%%%%%%%%%%%%%%%%%%%%%%%%%%%

\subsection{Distance Rationalizability View}
Meskanen and Nurmi~\cite{MN08} introduced the distance rationalizability framework where given a consensus class of profiles where the output is already defined and a distance function between profiles, the corresponding distance rationalizable rule, given any profile, finds the closest profile in the consensus class and returns the output defined on that. Edith et. al.~\cite{EFS10} study additively votewise distances, where the distance between two profiles is (vaguely) the sum of distances between their rankings. We note that the additively votewise distance rationalizable rules with (strong) unanimity consensus class and the distance between two rankings being square of the Euclidean distance ($\ell_2$ norm) between their embeddings are exactly symmetric mean proximity. 

Note that while the Euclidean distance defines a distance metric (satisfying triangle inequality), its square may not be a distance metric. However, it turns out that for the pairwise comparison embedding that generates the Kemeny rule, the square of Euclidean distance is the Kendall Tau distance (up to a multiplicative constant), which is a distance metric. This has two important implications. First, it shows that under the more natural Euclidean distance function, Kemeny is indeed a \emph{mean} rule rather than a \emph{median} rule. This explains why the Kemeny rule has desirable properties, as the importance of mean consensus has been emphasized significantly in the literature. Second, it indicates that the distance metrics whose square are also distance metrics might be of independent interest, and might be connected to rules satisfying desirable properties. Another example of a distance metric whose square is also a distance metric is square root of the Hamming distance.

%%%%%%%%%%%%%%%%%%%%%%%%%%%%%%%%%%%%%%%%%%%%%

\subsection{MLE View}
Given an embedding $\phi$, consider the noise model where 
$$
\Pr[\sigma | \sigma^*] = \frac{e^{-\|\phi(\sigma)-\phi(\sigma^*)\|^2}}{\sum_{\sigma' \in \rank} e^{-\|\phi(\sigma')-\phi(\sigma^*)\|^2}} \propto e^{-\|\phi(\sigma)-\phi(\sigma^*)\|^2}.
$$

This is a family of Gaussian noise models. Note that Mallows' model is a special case with $\phi$ being the pairwise comparison embedding. This indicates that Mallows' model is actually member of a Gaussian family. Further, define a distance function $d(\sigma,\sigma') = \|\phi(\sigma)-\phi(\sigma')\|$. When $d$ is neutral, i.e., $d(\tau \sigma,\tau \sigma') = d(\sigma,\sigma')$, the normalization is independent of $\sigma^*$. Note that this is indeed the case with linear embeddings. 

There are two interesting questions in this domain.
\begin{enumerate}
\item Mallows' model with the Kendall Tau distance admits an efficient sampling procedure. Is that the case with the family of Gaussian models introduced above when the embedding is linear? 
\item When we take the embedding for a positional scoring rule, we get a distribution where the probability of a ranking decreases exponentially as its discrepancy from the true ranking increases, and the discrepancy is measured by positions of alternatives rather than pairwise comparisons. Is there anything interesting about this distribution like Mallows' model?
\end{enumerate}

%%%%%%%%%%%%%%%%%%%%%%%%%%%%%%%%%%%%%%%%%%%%%
%%%%%%%%%%%%%%%%%%%%%%%%%%%%%%%%%%%%%%%%%%%%%
%%%%%%%%%%%%%%%%%%%%%%%%%%%%%%%%%%%%%%%%%%%%%

\section{Dimension of the Embeddings}
\label{sec:min-dim}

In this section, we analyze the following general question. \emph{Given a neutral symmetric mean proximity rule $r$, what is the minimum dimension (of the Euclidean space) required by any embedding representing $r$?}. For our main result of this section (Theorem~\ref{thm:dimension}), we need the following conjecture.
\begin{conjecture}
Let $r$ be any neutral symmetric mean proximity rule. Then, score matrices $S$ and $S'$ both represent $r$ if and only if there exist $a \in \mathbb{R}_{+}$ and $b \in \mathbb{R}^{m!}$ such that $S' = aS+B$, where $B = [b;\ldots;b]$ is the matrix where every row is $b$. 
\label{conj:equivalent-score-matrix}
\end{conjecture}
Note that all $aS+B$ transformations do preserve the underlying voting rule, because $(aS+B)y = a(Sy)+By$. Hence, $\argmax_{\sigma_i \in \rank} ((aS+B)y)_i = \argmax_{\sigma_i \in \rank} a(Sy)_i + by = \argmax_{\sigma_i \in \rank} (Sy)_i$. The other direction is still open.

\begin{theorem}
Let $S$ be any score matrix of a neutral symmetric mean proximity rule $r$. Let $c$ be the sum of elements of any row of $S$. Let $S'$ be the matrix such that $S'_{ij} = S_{ij}-c/{m!}$. Then, the minimum dimension required by any embedding representing $r$ is $rank(S')$. 
\label{thm:dimension}
\end{theorem}
\begin{proof}
\cns{Proved, to be written}
\end{proof}
%\begin{proof}
%Let $d$ be the minimum dimension achieved by any embedding representing $r$, and let $\phi$ be any such embedding. Let $B$ be the matrix whose rows are the $m!$ embeddings given by $\phi$. Then, by Equation~\eqref{eqn:equiv-dist-inner-product} in the proof of Theorem~\ref{thm:psd}, we know that $S' = B B^T$ is a score matrix of $r$. Also, $d \ge rank(B) = rank(B B^T = S')$. We first show that this inequality must in fact be an equality. This can be seen from the eigendecomposition of $S'$. Let $S' = U D U^T$. Then, $D$ is the diagonal matrix containing the eigenvalues. Only first $rank(S')$ diagonal entries are non-zero. Taking $B = U \sqrt{D}$ yeids an embedding of $r$, since $B B^T = U \sqrt{D} \sqrt{D} U^T = S'$. Further, last $m!-rank(S')$ columns of $B$ are zero. Hence, we can drop them to obtain an embedding with dimension $rank(S') \le d$. Since $d$ is the minimum dimension of any embedding representing $r$, we have that $d = rank(S')$.
%
%Now it is sufficient to show that for any score matrix $S$ of $r$, $rank(S') = rank([S;1])-1$. By Conjecture~\ref{conj:equivalent-score-matrix}, there exist $a \in \mathbb{R}_{+}$ and $b \in \mathbb{R}^{m!}$ such that $S' = aS+B$, where $B = [b;\ldots;b]$ is the matrix where every row is $b$. 
%
%
%Then, by fundamental properties of the rank of a matrix, we have $rank(S') \le rank([S;1])$. To see this, let $c_1,\ldots,c_{m!}$ represent the columns of $S$. Let $b = [b_1,\ldots,b_{m!}]$. Then, the columns of $S'$ are $a\cdot c_1+ b_1\cdot 1, \ldots, a\cdot c_{m!}+b_{m!}\cdot 1$. It is now easy to see the $column-space(S') \subseteq column-space([S;1])$, thus $rank(S') \le rank([S;1])$. 
%
%However, by Lemma~\ref{lem:neutral-mean-proximity-properties}, we have that the sum of all elements of every row of $S$ is the same, i.e., the sum of columns of $S$ is a multiple of the column vector $1$. Hence, $column-space(S) = column-space([S;1])$, i.e., $rank(S) = rank([S;1])$. Thus, we have $rank(S') \le rank(S)$. By symmetry, we can also achieve $rank(S) \le rank(S')$. Hence, we have that $rank(S) = rank(S')$. 
%
%Now, let $d$ be the score of (all) score matrices of $r$. Let $\phi$ be any embedding representing $r$ and let $d'$ be its dimension.
%\end{proof}

We investigate another connection. Let $\phi$ be any embedding of a neutral symmetric mean proximity rule $r$. Consider the embedding $\phi'(\sigma) = \phi(\sigma)-\phi_{avg}$, where $\phi_{avg} = (1/m!) \cdot \sum_{\sigma' \in \rank} \phi(\sigma')$. Then, the Gramian matrix of $\phi'$ (which is also a score matrix of $r$ since $\phi'$ embeds all rankings to points of unit norm) can be obtained from the score matrix of $\phi$ (the negation of the Euclidean distance matrix according to $\phi$) by the transformation described in Conjecture~\ref{conj:equivalent-score-matrix} in the unique way described in Theorem~\ref{thm:dimension}. 

For this, note that if $S$ and $S'$ are the negative Euclidean distance matrix and the Gramian matrix of $\phi$ and $\phi'$ respectively, then 
\begin{align*}
S_{ij} - 2 \cdot S'_{ij} &= -\|\phi(\sigma_i)-\phi(\sigma_j)\|^2 - 2 \cdot \langle \phi(\sigma_i)-\phi_{avg}, \phi(\sigma_j)-\phi_{avg}\rangle \\
&= -\|\phi(\sigma_i)\|^2 -\|\phi(\sigma_j)\|^2 + 2 \cdot \langle \phi(\sigma_i),\phi(\sigma_j) \rangle - 2 \cdot \langle \phi(\sigma_i),\phi(\sigma_j) \rangle \\
&\quad\quad + 2 \cdot \langle \phi(\sigma_i),\phi_{avg} \rangle + 2 \cdot \langle \phi(\sigma_j),\phi_{avg} \rangle - 2 \cdot \|\phi_{avg}\|^2 \\
&= -\left(\|\phi(\sigma_i)\|^2 - 2 \langle \phi(\sigma_i),\phi_{avg} + \|\phi_{avg}\|^2  \right)  \\
&\quad\quad - \left(\|\phi(\sigma_j)\|^2 - 2 \langle \phi(\sigma_j),\phi_{avg} + \|\phi_{avg}\|^2  \right) \\
&= - \|\phi(\sigma_i)-\phi_{avg}\|^2 - \|\phi(\sigma_j)-\phi_{avg}\|^2 = -2\cdot \lambda^2,
\end{align*}
where $\lambda = \|\phi(\sigma)-\phi_{avg}\|$ which is independent of $\sigma$ due to Lemma~\ref{lem:average-profile}. Thus, $S'_{ij} = (1/2) \cdot S_{ij} + \lambda^2$ for all $i,j$. Note that in the resulting matrix $S'$, the sum of any row $i$ is $\langle \phi'(\sigma_i),\phi'_{avg} \rangle$. However, $\phi'_{avg} = \phi_{avg}-\phi_{avg} = 0$. Hence, the sum of each row (and each column) is zero. Note that there is a unique way of obtaining a matrix with zero row and column sums via subtracting the same value from each element of the matrix, which is the transformation of Theorem~\ref{thm:dimension}. 

We just showed that the transformation of Theorem~\ref{thm:dimension}, when applied on a negative Euclidean distance matrix, is equivalent to the following three step transformation: i) take the embedding generating the matrix, ii) subtracting its average embedding from every embedding, iii) note that the final embedding has equal norm vectors, and take its Gramian matrix. Thus, we have the following alternative to Theorem~\ref{thm:dimension}: The minimum dimension required by any embedding representing a neutral symmetric mean proximity rule $r$ is equal to the rank of the Gramian matrix of any normalized embedding of $r$. 

Formally, let $r$ be any neutral symmetric mean proximity rule and $\phi$ be any embedding representing $r$. Let $\phi'$ be obtained by normalizing $\phi$. Then, the minimum dimension required by any embedding of $r$ is $rank(M)$, where $M_{ij} = \langle \phi'(\sigma_i), \phi'(\sigma_j) \rangle$ for any fixed enumeration $\sigma_1,\ldots,\sigma_{m!}$ of $\rank$. 


\cns{Interesting to consider: What are all possible normalized embeddings of a rule? It's not just rotations. Take two points on the circle within +/- 30 degree angle. Mapping them to any two points on that arc of the circle preserves the only neutral symmetric mean proximity rule: majority}

\begin{corollary}
The minimum dimension required by any embedding of a positional scoring rule over $m$ alternatives is $m-1$.
\end{corollary}
\begin{proof}
\cns{Proved, To be written.}
\end{proof}
%\begin{proof}
%Consider any positional scoring rule $r$ with arbitrary score vector $\alpha = [\alpha_1,\ldots,\alpha_m]$. Let $a_1,\ldots,a_m$ be any fixed enumeration of $A$. For any ranking $\sigma$ and $a \in A$, let $\sigma(a)$ denote the rank of $a$ in $\sigma$. Consider the positional embedding $\phi$ such that $\phi(\sigma) = [\alpha_{\sigma(a_1)},\ldots,\alpha_{\sigma(a_m)}]$ for all $\sigma \in \rank$. We already saw in Lemma~\ref{lem:symmetric-captures} that this corresponds to the positional scoring rule $r$. Let $B$ denote the matrix whose rows are embeddings of rankings according to $\phi$. Due to neutrality of positional scoring rules, $B B^T$ is a score matrix of $r$. Hence, the minimum dimension for this scoring rule is $rank(B B^T) = rank(B)$. 
%
%Now, note that $B$ has $m$ columns, and further by Lemma~\ref{lem:neutral-mean-proximity-properties}, the sum of all the columms is 
%\end{proof}

\begin{corollary}
The minimum dimension required by any embedding of the Kemeny rule over $m$ alternatives is $m\cdot(m-1)/2$.
\end{corollary}
\begin{proof}
\cns{Proved, To be written.}
\end{proof}

\subsection{Representations of the Borda Rule}
Zwicker~\cite{Zwicker08a} shows that the $m-1$ dimensional embedding for the Borda rule corresponds to a regular hexagon for $3$ alternatives. In fact, it is not too hard to observe (\cns{Already shown somewhere?}) that for $m$ alternatives, it is the regular polytope representing the permutahedron of the symmetric group $S_m$. It is also known (\cns{where?}) that the convex hull of the permutahedron is formed by connecting every ranking with all $m-1$ rankings at Kendall Tau distance $1$ from it (i.e., those obtained by a single adjacent swap). \emph{What interesting shapes do we get by replacing the Kendall Tau distance by other distances over rankings? What are the rules for such embeddings?}

%%%%%%%%%%%%%%%%%%%%%%%%%%%%%%%%%%%%%%%%%%%%%
%%%%%%%%%%%%%%%%%%%%%%%%%%%%%%%%%%%%%%%%%%%%%
%%%%%%%%%%%%%%%%%%%%%%%%%%%%%%%%%%%%%%%%%%%%%

%\section{Embeddings of the Borda Rule}
%In this section, we analyze the $m-1$ dimension embeddings of the Borda rule. 
%
%{\bf $\mathbf{3}$ alternatives:} Zwicker~\cite{Zwicker08a} informally demonstrated that for $3$ alternatives, any embedding that represents the Borda rule embeds the $6$ possible rankings at the corners of a regular hexagon as shown in Figure~\ref{fig:borda-3alt}. In fact, it is exactly the \emph{permutahedron} of the symmetric group $S_3$ where each ranking is connected to two rankings that are obtained by two possible adjacent swaps. Thus, neighbouring rankings have \emph{Kendall Tau} distance $1$ from each other. Note that there are hexagons that connect rankings at KT distance greater than $1$, but those do not correspond to the Borda rule.
%
%\begin{figure}
%\centering
%\begin{subfigure}[$3$ alternatives]
%  %\includegraphics[width=.4\linewidth]{Borda3}
%  {\rule{3cm}{3cm}}
%  \label{fig:borda-3alt}
%\end{subfigure}%
%\begin{subfigure}[$4$ alternatives]
%  %\includegraphics[width=.4\linewidth]{Borda4}
%  {\rule{3cm}{3cm}}
%  \label{fig:borda-4alt}
%\end{subfigure}%
%\caption{Embeddings for the Borda Rule}
%\label{fig:borda}
%\end{figure}
%
%{\bf $\mathbf{4}$ alternatives:} Interestingly, the observation that the Borda rule embeds rankings to the vertices of the regular polytope of the permutahedron carries over to the case of $4$ alternatives. Thus, each ranking is connected to $3$ rankings that are obtained by performing one of the three possible adjacent swaps. The polytope of the permutahedron of $S_4$, shown in Figure~\ref{fig:borda-4alt}, consists of $8$ hexagons and $6$ squares. Each hexagon contains the $6$ rankings that are obtained by either keeping the first alternative constant or the last alternative constant. Note that we can forget the alternative that is constant, and the hexagon exactly matches the embeddings of the rankings over the remaining $3$ alternatives. The $6$ squares each contain $4$ rankings that are obtained by swapping the first two or the last two alternatives. Thus, each square has the following form: 
%$$
%a \succ b \succ c \succ d \;\longrightarrow\; a \succ b \succ d \succ c \;\longrightarrow\; b \succ a \succ d \succ c \;\longrightarrow\; b \succ a \succ c \succ d.
%$$
%{\bf Replace this by a figure of a square, similarly draw a hexagon}
%
%For better understanding of permutahedrons, refer to Crisman~\cite{Crisman}. 
%
%

%%%%%%%%%%%%%%%%%%%%%%%%%%%%%%%%%%%%%%%%%%%%%
%%%%%%%%%%%%%%%%%%%%%%%%%%%%%%%%%%%%%%%%%%%%%
%%%%%%%%%%%%%%%%%%%%%%%%%%%%%%%%%%%%%%%%%%%%%

\section{Other Research Questions}
\begin{enumerate}
\item What shape, and what corresponding voting rule do we get if we replace the KT distance by some other distance in the permutahedron?
\item Proving a lower bound on the dimensions for the Kemeny rule.
\item Alternative proof of the characterization of positional scoring rules by observing that they are obtained by the representation $R$ where $R_{\tau}$ is the permutation matrix generated by $\tau$ (which permutes rankings according to the mapping $\sigma \rightarrow \tau \sigma$). (This representation and all possible initial embedding $\phi$). 
\item Conjecture by Conitzer et. al.~\cite{CRX09}: Consistent $+$ continuous $+$ neutral $\Leftrightarrow$ Neutral SRSF.
\item What are the equivalence classes of $\phi$ that lead to the same voting rule?
\item What about notions of consensus in Euclidean spaces other than mean $\rightarrow$ e.g., minimize maximum distance (everyone ``lets go'' equally)?
\end{enumerate}

\section{Scribbled Notes}
These are just high level intuitions. Need further investigations.
\begin{enumerate}
\item If the construction of Conitzer works then,
\begin{itemize}
\item We don't need equal diagonal in the theorem. We can take any PSD, do the construction, and obtain equivalent matrix with equal diagonals.
\item We can take the matrix in Lemma 6 (1) and apply that construction. If we prove that construction preserves PSD, then not only we get a matrix satisfying both (1) and (2), but also its Gramian decomposition will give us a neutral normalized embedding. 
\end{itemize}
\item We can approach the other way by constructing a neutral normalized embedding (enough to obtain any neutral embedding, normalization preserves it) and take its score matrix. 

\item We can prove that linear normalized embeddings are also enough. Normalization preserves linearity by the same representation since $R_{\tau} \phi_{avg} = \phi_{avg}$ must hold.

\item Neutral $\Rightarrow$ all rows permutations of each other $\Rightarrow$ sum = constant $\Leftrightarrow$ $[1,\ldots,1]$ is an eigenvector $\Leftrightarrow$ first coordinate of the embedding constant $\Leftrightarrow$ there is an embedding of rank-1

\item What is the permutation of all rankings generated by applying $\tau$ to each of them? What are all these $m!$ permutations?

\item Anything interesting about inner product maximizing rules that do not have embeddings of equal norm? Even symmetric ones may not be unanimous. 

\item An illustration: (I think) the \emph{Least Frequent Ranking Rule} is not a symmetric mean proximity rule. One score matrix is negative identity matrix. Can possibly show that no score matrix is PSD. Note that no $aS+B$ transformation can help.
\end{enumerate}


%%%%%%%%%%%%%%%%%%%%%%%%%%%%%%%%%%%%%%%%%%%%%
%%%%%%%%%%%%%%%%%%%%%%%%%%%%%%%%%%%%%%%%%%%%%
%%%%%%%%%%%%%%%%%%%%%%%%%%%%%%%%%%%%%%%%%%%%%

\begin{comment}
Mean proximity rule / generalized scoring rule / SRSF - Neutral $\Rightarrow$ SRSF iff MLE
{\bf Question:} (Linear) Mean Proximity Rules - Captures all ``pairwise comparison scoring rules''?
\end{comment}

%%%%%%%%%%%%%%%%%%%%%%%%%%%%%%%%%%%%%%%%%%%%%
%%%%%%%%%%%%%%%%%%%%%%%%%%%%%%%%%%%%%%%%%%%%%
%%%%%%%%%%%%%%%%%%%%%%%%%%%%%%%%%%%%%%%%%%%%%

\bibliographystyle{plain}
\bibliography{abbshort,ultimate}
\end{document}